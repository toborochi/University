\section{Definición Clásica de Probabilidad}
$$P(A)=\dfrac{n(A)}{n(S)}=\dfrac{\textrm{\# de casos favorables del Evento $A$}}{\textrm{\# total de casos posibles}}$$
\subsection{Probabilidad Condicionada}
Sean $A$ y $B$ eventos tal que $P(B)>0$ la probabilidad condicional de que ocurra $A$ dado que ha ocurrido el evento $B$ se denota por: $P(A/B)$. Se define como:
$$P(A/B)=\dfrac{P(A\cap B)}{P(B)}$$
Si $B$ es un evento tal que $P(B)>0$, entonces $P(\bullet/B)$ satisface los siguientes axiomas:
\begin{enumerate}[label=(\roman*)]
\item $0\leq P(A/B)\leq 1$
\item $P(S/B)=1$
\item $P\left( \bigcup\limits_{i=1}^{n} A_i/B\right)=
\displaystyle\sum_{i=1}^{n}P(A_i/B) \hspace{1cm}A_i \textrm{son eventos mutuamente excluyentes}$
\end{enumerate}
\subsubsection{Propiedades}
Si $B$ es un evento tal que $P(B)>0$, entonces $P(\bullet/B)$ tiene las siguientes propiedades:\blfootnote{En $P(\bullet/B)$ el símbolo $\bullet$ significa \textit{cualesquiera}}
\begin{enumerate}
\item $P(\phi/B)=0$
\item $P(A^c/B)=1-P(A/B)\hspace{0.5cm} \vee \hspace{0.5cm} P(A/B)=1-P(A^c/B) $
\item $P((A\cup C)/B)=P(A/B)+P(C/B)-P((A\cap C)/B)$
\item $A\in C \Rightarrow P(A/B)\leq P(C/B)$
\end{enumerate}
\subsection{Eventos Dependientes}
Se dice que los eventos $A$ y $B$ son estadísticamente dependientes, si:

\begin{equation*}
\begin{rcases}
  P(A\cap B)=P(B)\cdot P(A/B)\\
  P(A\cap B)=P(A)\cdot P(B/A)
\end{rcases}
\text{Regla General de la Multiplicación}
\end{equation*}
\subsection{Eventos Independientes}
Se dice que los eventos $A$ y $B$ son estadísticamente independientes, si:
$$P(A\cap B)=P(A)\cdot P(B)$$
\subsubsection{Observaciones}
\begin{enumerate}
\item $\begin{rcases}
  P(A)\\
  P(B)
\end{rcases} \text{Probabilidades Marginales}$
\item $\begin{rcases}
  P(A/B)\\
  P(B/A)
\end{rcases} \text{Probabilidades Condicionales}$
\item $P(A/B)\neq P(B/A)$
\item $P(A \cap B) \hspace{0.5cm} \text{Probabilidades Condicionales}$
\item Decir \textit{'Eventos Independientes'} no es lo mismo que decir \textit{'Evento Mutuamente Excluyente'}.
\end{enumerate}
\subsubsection{Teorema}
Si $A$ y $B$ son eventos mutuamente independientes, entonces:
\begin{enumerate}
\item $P(A/B)=P(A)$
\item $P(B/A)=P(B)$
\item $A^c$ y $B$ son independientes.
\item $A$ y $B^c$ son independientes.
\item $A^c$ y $B^c$ son independientes.
\end{enumerate}
\subsubsection{Demostración de los Teoremas}
\begin{itemize}
\item Demostración de $1.$ :
\begin{align*}
P(A/B) &\varstackrel{1}{=} \dfrac{P(A\cap B)}{P(B)} \\
&\varstackrel{2}{=} \dfrac{P(A)\cdot P(B)}{P(B)} \\
&\varstackrel{3}{=} P(A)
\end{align*}
\item Demostración de $2.$ :
\begin{align*}
P(B/A) &\varstackrel{1}{=} \dfrac{P(B\cap A)}{P(A)} \\
&\varstackrel{2}{=} \dfrac{P(A)\cdot P(B)}{P(A)}  \\
&\varstackrel{3}{=} P(B)
\end{align*}
\item Demostración de $3.$ :
$$P(A^c \cap B) = P(A^c)\cdot P(B)$$
Formamos el siguiente sistema:
$$
\begin{cases} 
B = (A\cap B) \cup (A^c \cap B)\\
\phi = (A\cap B) \cap (A^c \cap B)
\end{cases}
$$
Ya que son excluyentes:
$$P(B)=P(A\cap B)+P(A^c \cap B)$$
Despejamos $P(A^c \cap B)$ ya que es lo que nos interesa demostrar:
$$P(A^c \cap B)=P(B)-P(A\cap B)$$
Como $A$ y $B$ son independientes, según la hipótesis, tenemos que $P(A\cap B)=P(A)P(B)$\\
Reemplazando en el anterior paso:
\begin{align*}
P(A^c \cap B) &\varstackrel{1}{=} P(B)-P(A)P(B) \\
&\varstackrel{2}{=} P(B)\cdot [1-P(A)]  \\
&\varstackrel{3}{=} P(B)\cdot P(A^c)
\end{align*}
Con esto que demostrado $3$.
\item Demostración de $4.$ :
$$P(A \cap B^c) = P(A)\cdot P(B^c)$$
Formamos el siguiente sistema:
$$
\begin{cases} 
A = (A\cap B) \cup (A \cap B^c)\\
\phi = (A\cap B) \cap (A\cap B^c)
\end{cases}
$$
Ya que son excluyentes:
$$P(A)=P(A\cap B^c)+P(A \cap B)$$
Despejamos $P(A \cap B^c)$ ya que es lo que nos interesa demostrar:
$$P(A \cap B^c)=P(A)-P(A\cap B)$$
Como $A$ y $B$ son independientes, según la hipótesis, tenemos que $P(A\cap B)=P(A)P(B)$\\
Reemplazando en el anterior paso:
\begin{align*}
P(A \cap B^c) &\varstackrel{1}{=} P(A)-P(A)P(B) \\
&\varstackrel{2}{=} P(A)\cdot [1-P(B)]  \\
&\varstackrel{3}{=} P(A)\cdot P(B^c)
\end{align*}
Con esto queda demostrado $4$.
\item Demostración de $5.$ :
En este punto, lo que tratamos de demostrar es: 
$$P(A^c \cap B^c)=P(A^c)P(B^c)$$
Comenzamos aplicando D'Morgan:
\begin{align*}
P(A^c \cap B^c) &\varstackrel{1}{=} P[(A\cup B)^c] \\
&\varstackrel{2}{=} 1-P(A\cup B) \\
&\varstackrel{3}{=} 1-[P(A)+P(B)-P(A\cap B)]\\
&\varstackrel{4}{=} 1-P(A)-P(B)+P(A\cap B) \\
&\varstackrel{5}{=} [1-P(A)]-[P(B)-P(A\cap B)]\\ 
&\varstackrel{6}{=} [1-P(A)]-[P(B)-P(A)P(B)]\\
&\varstackrel{7}{=} [1-P(A)]-P(B)[1-P(A)]\\
&\varstackrel{8}{=} [1-P(A)][1-P(B)]\\
&\varstackrel{9}{=} P(A^c)P(B^c)
\end{align*}
Con esto queda demostrado $5$.
\end{itemize}
%$\mathscr{L}$
\subsection{Teorema de la Probabilidad Total}
También llamada \textit{Regla de Eliminación}. Si los eventos mutuamente excluyentes:
$$B_1,B_2,B_3,\ldots,B_n$$
Constituyen una partición del espacio muestral $S$, de tal manera que $\forall i\in I,P(B_i)>0$. Entonces para cualquier evento $A$ de $S$ se cumple:
$$P(A)=\displaystyle\sum_{i=1}^{n} P(B_i)P(A/B_i)$$
\subsubsection{Demostración}
\begin{align*}
P(A) &\varstackrel{1}{=} [(B_1\cap A)\cup (B_2\cap A)\cup \ldots \cup (B_n\cap A)] \\
&\varstackrel{2}{=} P(B_1\cap A)+P(B_2\cap A)+ \ldots + P(B_n\cap A) \\
&\varstackrel{3}{=} \displaystyle\sum_{i=1}^{n}(B_i \cap A) \\
&\varstackrel{4}{=} \displaystyle\sum_{i=1}^{n} P(B_i)\cdot P(A/B_i) 
\end{align*}
\subsection{Teorema de Bayes}
Si los eventos: $B_1,B_2,B_3,\ldots,B_n$ , constituyen una partición del espacio muestral $S$, de tal modo que; $\forall i\in I,P(B_i)>0$, entonces para cualquier evento $A$ de $S$, tal que $P(A)>0$ se cumple:
$$P(A)=\dfrac{P(B_r)P(A/B_r)}{\displaystyle\sum_{i=1}^{n} P(B_i)P(A/B_i)}$$
\subsubsection{Demostración}
\begin{align*}
P(B_r/A) &\varstackrel{1}{=} \dfrac{P(B_r\cap A)}{P(A)} \\
&\varstackrel{2}{=} \dfrac{P(B_r)P(A/B_r)}{P(A)}  \\
&\varstackrel{3}{=} \dfrac{P(B_r)P(A/B_r)}{\displaystyle\sum_{i=1}^{n} P(B_i)P(A/B_i)}
\end{align*}
\section{Ejemplos}
\begin{enumerate}
\item Un inversionista planea escoger dos de las cinco oportunidades de inversión que le han recomendado. Describa el espacio muestral que represeta las opciones posibles.
\item Tres artículos son extraídos con reposición de un lote de mercancías, cada artículo ha de ser identificado como defectuoso (D) y no defectuoso (N). Describir todos los puntos posibles del espacio muestral para este experimento.
\item Una moneda se lanza tres veces. Describir los siguientes eventos:
\begin{itemize}
\item $A:$ \textit{Ocurre por lo menos dos caras.}
\item $B:$ \textit{Ocurre, sello en el tercer lanzamiento.}
\item $C:$ \textit{Ocurre a lo mas, una cara.}
\end{itemize}
Luego hallar:
\begin{enumerate}
\item $A\cup B$
\item $B-C$
\item $A^c \cap B^c$
\item $A\triangle C$
\end{enumerate}
\subsubsection{Solución}
\item Sea el experimento, \textit{Lanzar una moneda hasta que ocurra cara y contar el número de lanzamientos de la moneda.}  Considerando:
\begin{itemize}
\item $A:$ \textit{Se necesita un número par de lanzamientos}
\item $B:$ \textit{Se necesitan mas de diez lanzamientos.}
\end{itemize}
Hallar:
\begin{enumerate}
\item $A\cap B$
\item $A-B$
\item $B-A$
\item $A^c$
\item $B^c$
\item $A^c \cap B^c$
\end{enumerate}
\subsubsection{Solución}
\item Demostrar que: $P(A/B)+P(A^c/B)=1$
\subsubsection{Solución}
\begin{align*}
P(A/B)+P(A^c/B) &\varstackrel{1}{=} \dfrac{P(A\cap B)}{P(B)} + \dfrac{P(A^c\cap B)}{P(B)} \\
  &\varstackrel{2}{=} \dfrac{P(A\cap B)+P(A^c\cap B)}{P(B)} \\
    &\varstackrel{3}{=} \dfrac{P[(A\cap B)\cup(A^c\cap B)]}{P(B)} \\
      &\varstackrel{4}{=} \dfrac{P[(A\cup A^c)\cap B]}{P(B)} \\
        &\varstackrel{5}{=} \dfrac{P(\Omega\cap B)}{P(B)} \\
           &\varstackrel{6}{=} \dfrac{P(B)}{P(B)} \\
                 &\varstackrel{7}{=} 1 \\
\end{align*}
$\therefore P(A/B)+P(A^c/B)=1$
\item Demostrar que: $P(A/B^c)+P(A^c/B)=1$
\subsubsection{Solución}
\begin{align*}
P(A/B^c)+P(A^c/B) &\varstackrel{1}{=} \dfrac{P(A\cap B^c)}{P(B^c)} + \dfrac{P(A^c\cap B)}{P(B)} \\
&\varstackrel{2}{=} \dfrac{P(A)-P(A\cap B)}{P(B^c)}+ \dfrac{P(B)-P(A\cap B)}{P(B)}
\end{align*}
\item La probabilidad de que llueva en Santa Cruz el 10 de Junio es de 40\%, de que truene es 5\% y de que, llueve y truene es de 3\%. ¿Cual es la probabilidad de que llueva o truene ese día?
\subsubsection{Solución}
Conociendo los siguientes eventos:
\begin{itemize}
\item $A:$ Llueva
\item $B:$ Truene
\item $A\cap B:$ Llueva y truene
\end{itemize}
Sabemos los valores de las probabilidades de cada uno, de acuerdo al enunciado:
\begin{itemize}
\item $P(A)=40\%$
\item $P(B)=5\%$
\item $P(A\cap B)=3\%$
\end{itemize}
Con estos datos nos queda hallar $A\cup B$, por lo tanto calculamos simplemente reemplazando en la igualdad:
\begin{align*}
 P(A\cup B) &= P(A)+P(B)-P(A\cap B)\\
 	      &= 40\% + 5\% - 3\%    \\
 	      	&= 42\%
\end{align*}
\item Sea $A$ y $B$ dos eventos, tal que: $P(A)=0.20\% ; P(B)=0.30\%;P(A\cup B)=0.10\%$ \\ Hallar:
\begin{enumerate}
\item $P(A^c\cap B^c)$
\item $P(A^c\cap B)$
\item $P(A\cap B^c)$
\item $P(A^c\cup B)$
\end{enumerate}


\subsubsection{Solución}
\item Con 7 ingenieros y 4 médicos se formaron comités de 6 miembros. ¿Cuál es la probabilidad que el comité incluya: ?
\begin{enumerate}
\item Exactamente 2 médicos.
\item Al menos 2 ingenieros.
\end{enumerate}
\subsubsection{Solución}
\item En la UAGRM el 30\% de los estudiantes son cruceños, el 10\% estudia Ingeniería Informática, y el 1\% son cruceños y estudian Informática. Si se selecciona al azar un estudiante de la UAGRM. Hallar las siguientes probabilidades:
\begin{enumerate}
\item El estudiante no es cruceño.
\item Sea cruceño o pertenezca a Ingeniería Informática.
\item Sea cruceño y no estudie Ingeniería Informática.
\item No sea cruceño ni estudie Ingeniería Informática.
\end{enumerate}
\subsubsection{Solución}
\item En una encuesta pública se determina que la probabilidad que una persona consuma $A$ es $0.50$ que consuma $B$ es $0.37$, que consuma $C$ es $0.30$, que consuma $A$ y $B$ es $0.12$ que consuma solamente $A$ y $C$ es $0.08$ que consuma solamente $B$ y $C$ es $0.05$ y que consuma solamente $C$ es $0.15$.
\begin{itemize}
\item Calcular la probabilidad que de que alguien consuma $A$ o $B$ pero no $C$.
\end{itemize}
\subsubsection{Solución}
\item Se lanza un dado legal sobre una mesa y se observa el número que aparece en la cara superior.
\begin{enumerate}
\item ¿Cual es la probabilidad de obtener un número par?
\item ¿Cual es la probabilidad de obtener un número impar?
\item Los eventos \textit{''Obtener número par''} y \textit{''Obtener número impar''} son mutuamente excluyentes?¿Son Independientes?
 
\end{enumerate}
\subsubsection{Solución}
\begin{enumerate}
\item Primeramente definamos el espacio muestral, que sería $\Omega = \lbrace 1,2,3,4,5,6 \rbrace$, de este e.m. tomamos en cuenta todos los casos que sean favorables al enunciado, \textit{''Obtener Número Par''}, estos son un evento $A=\lbrace 2,4,6\rbrace$. Finalmente:
$$P(A)=\dfrac{n(A)}{n(\Omega)}=\dfrac{3}{6}=\dfrac{1}{2} (50\%)$$
\item Bajo la misma lógica anterior realizamos para los números impares, teniendo: $B=\lbrace 1,3,5\rbrace$
$$P(B)=\dfrac{n(B)}{n(\Omega)}=\dfrac{3}{6}=\dfrac{1}{2} (50\%)$$
\end{enumerate}
\item Una urna contiene 6 fichas blancas y 4 negras. Se extraen 2 fichas sucesivamente y sin restitución.
\begin{enumerate}
\item ¿Cual es la probabilidad de que ambas sean negras?
\item ¿Cual es la probabilidad de que la primera sea blanca y la segunda negra?
\item ¿Cual es la probabilidad de que la primera sea negra y la segunda blanca?
\item ¿Cual es la probabilidad de que ambas sean blancas?
\end{enumerate}
\item En un grupo de personas hay 3 mujeres y 4 hombres varones. Si se elige una persona al azar ¿Cual es la probabilidad que sea varón?
\item Si $A$ y  $B$ son eventos cualesquiera, demostrar:
$$P(A\cup B\cup C)=P(A)+P(B)+P(C)-P(A\cap C)-P(B\cap C)-P(A\cap B)+P(A\cap B \cap C)$$
\subsubsection{Solución}
\item La probabilidad de que un estudiante apruebe matemáticas es $\frac{2}{3}$ y que apruebe física es $\frac{4}{9}$. Si la probabilidad de aprobar al menos una de estas materias es $\frac{4}{5}$. ¿ Cual es la probabilidad de aprobar ambas materias?
\subsubsection{Solución}
El problema nos plantea los siguientes datos:
\begin{itemize}
\item $A:$ \textit{Aprobar Matemáticas.}
\item $B:$ \textit{Aprobar Física.}
\item $A\cup B :$ \textit{Aprobar al menos una de estas materias.} (Es decir, aprobar Física o Matemáticas)
\end{itemize}
Por lo tanto, las probabilidades ya están dadas:
\begin{itemize}
\item $P(A)= \frac{2}{3}$
\item $P(B)=  \frac{4}{9}$
\item $P(A\cup B)= \frac{4}{5}$
\end{itemize}
Como el problema pide la probabilidad de aprobar ambas materias, esto es: $P(A\cap B)$. Si partimos de la siguiente igualdad:
\begin{align*}
P(A\cup B) = P(A) + P(B) - P(A\cap B)
\end{align*}
Luego, despejamos $P(A\cap B)$ y reemplazamos:
\begin{align*}
P(A\cap B) &= P(A) + P(B) - P(A\cup B) \\
           &= \dfrac{2}{3}+\dfrac{4}{9}-\dfrac{4}{5} \\
           &= \dfrac{14}{45} \approx 31.11 \%
\end{align*}
Obtenemos que la probabilidad de aprobar ambas materias es de $31.11\%$.
\item En los últimos años la Universidad \textbf{ABC} ha estado llevando un registro de sus egresados, en la actualidad tienen empleo, anotando el número de años que utilizaron para terminar su carrera y su posición (alta, media o baja) que tienen como profesionales.
\begin{center}
\begin{tabular}{|c|c|c|c|c|}
  \hline
  \diagbox[innerwidth=2cm,height=1.2cm]{Posición}{Tiempo} & Alta & Media & Baja  & Total \\ \hline
  5 Años & $30$ & $70$ & $20$ & $120$ \\ \hline
  +5 Años & $20$ & $30$ & $30$ & $80$ \\ \hline
  Total & $50$ & $100$ & $50$ & $200$ \\
  \hline
\end{tabular}
\end{center}
\begin{enumerate}
\item Basados en la información anterior.  ¿ Cual es la probabilidad de que si la duración de sus estudios fue de 5 años, tenga una alta posición profesional en su empleo actual?
\item Si el empleado tiene baja posición ¿Cual es la probabilidad de que tal persona haya realizado sus estudios en +5 años?
\item Si el egresado tiene posición profesional media, ¿ Cual es la probabilidad de que tal persona haya realizado la cerrera en 5 años?
\end{enumerate}
\item La urna $A$ contiene 6 fichas grises y 4 rojas, y la urna $B$ contiene 2 fichas grises y 7 rojas. Se saca una ficha de $A$ y se coloca en la $B$, en seguida se saca una ficha de $B$. Dado que la ficha extraída de $B$ es gris. ¿Cual es la probabilidad de que la ficha sacada de $A$ sea gris?
\item La urna $A_1$ contiene 8 fichas blancas y 2 negras. La urna $A_2$ contiene 3 fichas blancas y 7 negras, finalmente, la urna $A_3$ contiene 5 fichas blancas y 5 negras. Se lanza un dado no cargado, si resulta $\left\lbrace 1,2,3\right\rbrace $ se saca una ficha de la urna $A_1$; si resulta $\left\lbrace 4,5 \right\rbrace $ se saca una ficha de $A_2$, si resulta $\left\lbrace 6 \right\rbrace $ se saca de la urna $A_3$. \\${ }$\\ Dado que la ficha extraída es blanca. ¿ Cual es la probabilidad de que venga de $A_2$?
\end{enumerate}
