%Ejemplos Capitulo 3
\section{Ejemplos}
\begin{enumerate}
\item La siguiente distribución se refiere a los salarios de cien empleados:
\subsubsection{Solución}
\item Las notas de sesenta alumnos en exámenes y trabajos prácticos son las siguientes:
\begin{enumerate}
\item Construir la distribución de frecuencias absoluta, conjuntas, relativas y acumuladas.
\item Calcular las medias de las distribuciones marginales.
\item Hallar la covarianza e interpretar.
\end{enumerate}
\subsubsection{Solución}
\item El siguiente cuadro es la distribución de 100 familias por el número de hijos $(x_i)$ y el número de dormitorios por vivienda $(y_i)$.
\begin{center}
\begin{tabular}{|c|c|c|c|}
  \hline
  \diagbox[innerwidth=1cm]{$x_i$}{$y_i$} & 1 & 2 & 3    \\ \hline
  0 & 9 & 4 & 0\\ \hline
  1 & 10 & 16 & 9 \\ \hline
  2 & 5 & 16 & 12\\ \hline
  3 & 0 & 6 & 13\\ 
  \hline
\end{tabular}
\end{center}
\begin{enumerate}
\item Construir la Distribución Marginal $x,y$
\item Construir las correspondientes distribuciones de frecuencia absoluta y relativa acumulada.
\item  ¿Cuantas familias tienen a lo sumo dos hijos y a lo sumo dos dormitorios?
\end{enumerate}
\subsubsection{Solución}
\end{enumerate}