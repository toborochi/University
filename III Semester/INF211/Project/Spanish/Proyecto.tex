\documentclass[10pt,letterpaper]{article}
\usepackage[utf8]{inputenc}
\usepackage[spanish]{babel}
\usepackage{amsmath}
\usepackage{amsfonts}
\usepackage{amssymb}
\usepackage{graphicx}
\usepackage{tikz}
\usetikzlibrary{matrix,calc}
\usepackage[left=2cm,right=2cm,top=2cm,bottom=2cm]{geometry}
\author{
\begin{tabular}{|c|c|}
\hline 
\textbf{Integrantes} & \textbf{Registros} \\ 
\hline 
Leonardo  Henry Añez Vladimirovna & 217002498 \\ 
\hline 
Erick Edwing Vidal Céspedes & 217055321 \\ 
\hline 
Cristian Coca Terceros & 217050662 \\ 
\hline 
\end{tabular} 
\vspace{4cm}
}
\title{
Universidad Autónoma Gabriél René Moreno \\
Facultad de Ingeniería en Ciencias de la  Computación y Telecomunicaciones \\\vspace{1cm}
\texttt{Arquitectura del Computador} \\\vspace{3cm}
Contador 0-99\\\vspace{2cm}}

\graphicspath{ {Imagenes/} }

%isolated term
%#1 - Optional. Space between node and grouping line. Default=0
%#2 - node
%#3 - filling color
\newcommand{\implicantsol}[3][0]{
    \draw[rounded corners=3pt, fill=#3, opacity=0.3] ($(#2.north west)+(135:#1)$) rectangle ($(#2.south east)+(-45:#1)$);
    }


%internal group
%#1 - Optional. Space between node and grouping line. Default=0
%#2 - top left node
%#3 - bottom right node
%#4 - filling color
\newcommand{\implicant}[4][0]{
    \draw[rounded corners=3pt, fill=#4, opacity=0.3] ($(#2.north west)+(135:#1)$) rectangle ($(#3.south east)+(-45:#1)$);
    }

%group lateral borders
%#1 - Optional. Space between node and grouping line. Default=0
%#2 - top left node
%#3 - bottom right node
%#4 - filling color
\newcommand{\implicantcostats}[4][0]{
    \draw[rounded corners=3pt, fill=#4, opacity=0.3] ($(rf.east |- #2.north)+(90:#1)$)-| ($(#2.east)+(0:#1)$) |- ($(rf.east |- #3.south)+(-90:#1)$);
    \draw[rounded corners=3pt, fill=#4, opacity=0.3] ($(cf.west |- #2.north)+(90:#1)$) -| ($(#3.west)+(180:#1)$) |- ($(cf.west |- #3.south)+(-90:#1)$);
}

%group top-bottom borders
%#1 - Optional. Space between node and grouping line. Default=0
%#2 - top left node
%#3 - bottom right node
%#4 - filling color
\newcommand{\implicantdaltbaix}[4][0]{
    \draw[rounded corners=3pt, fill=#4, opacity=0.3] ($(cf.south -| #2.west)+(180:#1)$) |- ($(#2.south)+(-90:#1)$) -| ($(cf.south -| #3.east)+(0:#1)$);
    \draw[rounded corners=3pt, fill=#4, opacity=0.3] ($(rf.north -| #2.west)+(180:#1)$) |- ($(#3.north)+(90:#1)$) -| ($(rf.north -| #3.east)+(0:#1)$);
}

%group corners
%#1 - Optional. Space between node and grouping line. Default=0
%#2 - filling color
\newcommand{\implicantcantons}[2][0]{
    \draw[rounded corners=3pt, opacity=.3] ($(rf.east |- 0.south)+(-90:#1)$) -| ($(0.east |- cf.south)+(0:#1)$);
    \draw[rounded corners=3pt, opacity=.3] ($(rf.east |- 8.north)+(90:#1)$) -| ($(8.east |- rf.north)+(0:#1)$);
    \draw[rounded corners=3pt, opacity=.3] ($(cf.west |- 2.south)+(-90:#1)$) -| ($(2.west |- cf.south)+(180:#1)$);
    \draw[rounded corners=3pt, opacity=.3] ($(cf.west |- 10.north)+(90:#1)$) -| ($(10.west |- rf.north)+(180:#1)$);
    \fill[rounded corners=3pt, fill=#2, opacity=.3] ($(rf.east |- 0.south)+(-90:#1)$) -|  ($(0.east |- cf.south)+(0:#1)$) [sharp corners] ($(rf.east |- 0.south)+(-90:#1)$) |-  ($(0.east |- cf.south)+(0:#1)$) ;
    \fill[rounded corners=3pt, fill=#2, opacity=.3] ($(rf.east |- 8.north)+(90:#1)$) -| ($(8.east |- rf.north)+(0:#1)$) [sharp corners] ($(rf.east |- 8.north)+(90:#1)$) |- ($(8.east |- rf.north)+(0:#1)$) ;
    \fill[rounded corners=3pt, fill=#2, opacity=.3] ($(cf.west |- 2.south)+(-90:#1)$) -| ($(2.west |- cf.south)+(180:#1)$) [sharp corners]($(cf.west |- 2.south)+(-90:#1)$) |- ($(2.west |- cf.south)+(180:#1)$) ;
    \fill[rounded corners=3pt, fill=#2, opacity=.3] ($(cf.west |- 10.north)+(90:#1)$) -| ($(10.west |- rf.north)+(180:#1)$) [sharp corners] ($(cf.west |- 10.north)+(90:#1)$) |- ($(10.west |- rf.north)+(180:#1)$) ;
}

%Empty Karnaugh map 4x4
\newenvironment{Karnaugh}%
{
\begin{tikzpicture}[baseline=(current bounding box.north),scale=0.8]
\draw (0,0) grid (4,4);
\draw (0,4) -- node [pos=1.3,above right,anchor=south west] {$Q_3 Q_2$} node [pos=0.7,below left,anchor=north east] {$Q_1 Q_0$} ++(135:1);
%
\matrix (mapa) [matrix of nodes,
        column sep={0.8cm,between origins},
        row sep={0.8cm,between origins},
        every node/.style={minimum size=0.3mm},
        anchor=8.center,
        ampersand replacement=\&] at (0.5,0.5)
{
                       \& |(c00)| 00         \& |(c01)| 01         \& |(c11)| 11         \& |(c10)| 10         \& |(cf)| \phantom{00} \\
|(r00)| 00             \& |(0)|  \phantom{0} \& |(1)|  \phantom{0} \& |(3)|  \phantom{0} \& |(2)|  \phantom{0} \&                     \\
|(r01)| 01             \& |(4)|  \phantom{0} \& |(5)|  \phantom{0} \& |(7)|  \phantom{0} \& |(6)|  \phantom{0} \&                     \\
|(r11)| 11             \& |(12)| \phantom{0} \& |(13)| \phantom{0} \& |(15)| \phantom{0} \& |(14)| \phantom{0} \&                     \\
|(r10)| 10             \& |(8)|  \phantom{0} \& |(9)|  \phantom{0} \& |(11)| \phantom{0} \& |(10)| \phantom{0} \&                     \\
|(rf) | \phantom{00}   \&                    \&                    \&                    \&                    \&                     \\
};
}%
{
\end{tikzpicture}
}

%Empty Karnaugh map 2x4
\newenvironment{Karnaughvuit}%
{
\begin{tikzpicture}[baseline=(current bounding box.north),scale=0.8]
\draw (0,0) grid (4,2);
\draw (0,2) -- node [pos=0.7,above right,anchor=south west] {bc} node [pos=0.7,below left,anchor=north east] {a} ++(135:1);
%
\matrix (mapa) [matrix of nodes,
        column sep={0.8cm,between origins},
        row sep={0.8cm,between origins},
        every node/.style={minimum size=0.3mm},
        anchor=4.center,
        ampersand replacement=\&] at (0.5,0.5)
{
                      \& |(c00)| 00         \& |(c01)| 01         \& |(c11)| 11         \& |(c10)| 10         \& |(cf)| \phantom{00} \\
|(r00)| 0             \& |(0)|  \phantom{0} \& |(1)|  \phantom{0} \& |(3)|  \phantom{0} \& |(2)|  \phantom{0} \&                     \\
|(r01)| 1             \& |(4)|  \phantom{0} \& |(5)|  \phantom{0} \& |(7)|  \phantom{0} \& |(6)|  \phantom{0} \&                     \\
|(rf) | \phantom{00}  \&                    \&                    \&                    \&                    \&                     \\
};
}%
{
\end{tikzpicture}
}

%Empty Karnaugh map 2x2
\newenvironment{Karnaughquatre}%
{
\begin{tikzpicture}[baseline=(current bounding box.north),scale=0.8]
\draw (0,0) grid (2,2);
\draw (0,2) -- node [pos=0.7,above right,anchor=south west] {b} node [pos=0.7,below left,anchor=north east] {a} ++(135:1);
%
\matrix (mapa) [matrix of nodes,
        column sep={0.8cm,between origins},
        row sep={0.8cm,between origins},
        every node/.style={minimum size=0.3mm},
        anchor=2.center,
        ampersand replacement=\&] at (0.5,0.5)
{
          \& |(c00)| 0          \& |(c01)| 1  \\
|(r00)| 0 \& |(0)|  \phantom{0} \& |(1)|  \phantom{0} \\
|(r01)| 1 \& |(2)|  \phantom{0} \& |(3)|  \phantom{0} \\
};
}%
{
\end{tikzpicture}
}

%Defines 8 or 16 values (0,1,X)
\newcommand{\contingut}[1]{%
\foreach \x [count=\xi from 0]  in {#1}
     \path (\xi) node {\x};
}

%Places 1 in listed positions
\newcommand{\minterms}[1]{%
    \foreach \x in {#1}
        \path (\x) node {1};
}

%Places 0 in listed positions
\newcommand{\maxterms}[1]{%
    \foreach \x in {#1}
        \path (\x) node {0};
}

%Places X in listed positions
\newcommand{\indeterminats}[1]{%
    \foreach \x in {#1}
        \path (\x) node {X};
}

\begin{document}

\maketitle

\pagebreak

\section*{Introducción}
El contador digital es aquella secuencia constante de números de 0 a 9, el cual se aprecia su funcionamiento en un display de 7 Segmentos, en este proyecto veremos el
funcionamiento lógico de los circuitos internos utilizados para llegar a que funcione este contador.
\section{Descripción}
Para el circuito, desarrollaremos las tablas de Karnaugh para el caso 0-9 (Contador \texttt{MOD10}), ya que de este se puede partir a una generalización de un contador.
\subsection{Decodificador BCD de 7 Segmentos (7447)}
Es un dispositivo que \textit{decodifica} un código de entrada en otro. Es decir, transforma una combinación de unos y cero, en otra. 74LS47, en particular transforma el código binario en el código de 7 segmentos.
\begin{center}
\includegraphics[scale=0.5]{7447}
\end{center}
El decodificador recibe en su entrada el número que será visualizado en el display. Posee 7 salidas, una para cada segmento. Para un valor de entrada, cada salida toma un estado determinado (activada o desactivada).
\subsubsection{Uso del  Decodificador}
Entonces, como ya lo dijimos, hay que aplicar el número deseado en la entrada y el dispositivo, automáticamente, habilita los segmentos correspondientes a la salida. Supongamos que queremos mostrar el número 5. Utilizando la tabla anterior vemos que 5 en binario es \texttt{0101.} Debemos aplicar este valor en los pines de entrada en el orden \texttt{DCBA}, es decir \texttt{DCBA = 0101}, o sea \texttt{D=0, C=1, B=0, A=1.} Al hacerlo, el integrado enciende todos los segmentos, salvo \texttt{b} y \texttt{e} para mostrar el número 5.
\begin{center}
\includegraphics[scale=0.75]{Ejemplo7447}
\end{center}
\section{Materiales}
\begin{itemize}
\item Circuito Integrado \texttt{555}
\item Circuitos Integrados \texttt{SN7447AN} (2)
\item Circuitos Integrados \texttt{SN74SL90N} (2)
\item Resistencia de 220 Ohm (14)
\item Display de 7 Segmentos, Anodo Comun (2)
\item Protoboard (2)
\item 1 metro de Cable \texttt{UTP}
\item Potenciometro 10K (1)
\item Condensador de 100$\mu$F
\item Fuente de Alimentación de \texttt{5V}
\item Multímetro Básico (Instrumentación)
\item Pinzas (Instrumentación)
\end{itemize}
\section{Diseño Lógico}
Estados del \texttt{Contador MOD10}
\begin{center}
\begin{tabular}{|c|c|c|c|c|}
\hline 
\texttt{DEC} & $Q_3$ & $Q_2$ & $Q_1$ & $Q_0$ \\ 
\hline 
0 & 0 & 0 & 0 & 0 \\ 
\hline 
1 & 0 & 0 & 0 & 1 \\ 
\hline 
2 & 0 & 0 & 1 & 0 \\ 
\hline 
3 & 0 & 0 & 1 & 1 \\ 
\hline 
4 & 0 & 1 & 0 & 0 \\ 
\hline 
5 & 0 & 0 & 0 & 1 \\ 
\hline 
6 & 0 & 1 & 1 & 0 \\ 
\hline 
7 & 0 & 1 & 1 & 1 \\ 
\hline 
8 & 1 & 0 & 0 & 0 \\ 
\hline 
9 & 1 & 0 & 0 & 1 \\ 
\hline 
\end{tabular} 
\end{center}

En esta sección se detallan los mapas de Karnaugh para el \texttt{Contador MOD10 (0-9)}

\subsubsection{Mapas de Karnaugh}

\begin{center}
    \begin{Karnaugh}
        \contingut{1,1,1,\textbf{x},x,x,x,\textbf{x},1,1,\textbf{x},\textbf{x},x,x,\textbf{x},\textbf{x}}
    \implicant{0}{10}{green}
    \end{Karnaugh}
    %
   \begin{Karnaugh}
        \contingut{x,x,x,\textbf{x},1,1,1,\textbf{x},x,x,\textbf{x},\textbf{x},1,1,\textbf{x},\textbf{x}}
    \implicant{0}{10}{purple}
    \end{Karnaugh}   
\end{center}


\begin{center}
    \begin{Karnaugh}
        \contingut{0,0,0,\textbf{x},1,1,0,\textbf{x},x,x,\textbf{x},\textbf{x},x,x,\textbf{x},\textbf{x}}
     \implicant{4}{13}{yellow}
    \end{Karnaugh}
    %
   \begin{Karnaugh}
        \contingut{x,x,x,\textbf{x},x,x,x,\textbf{x},0,0,\textbf{x},\textbf{x},1,1,\textbf{x},\textbf{x}}
    \implicant{4}{14}{red}
    \end{Karnaugh}
\end{center}


\begin{center}
    \begin{Karnaugh}
        \contingut{0,x,0,\textbf{x},0,x,0,\textbf{x},0,x,\textbf{x},\textbf{x},1,x,\textbf{x},\textbf{x}}
  	    \implicant{12}{14}{blue}
    \end{Karnaugh}
    %
   \begin{Karnaugh}
        \contingut{x,0,0,\textbf{x},x,0,x,\textbf{x},x,0,\textbf{x},\textbf{x},x,1,\textbf{x},\textbf{x}}
	    \implicant{12}{14}{pink}
    \end{Karnaugh}
\end{center}


\begin{center}
    \begin{Karnaugh}
        \contingut{0,0,x,\textbf{x},0,0,x,\textbf{x},0,0,\textbf{x},\textbf{x},0,1,\textbf{x},\textbf{x}}
  	    \implicant{13}{15}{orange}
    \end{Karnaugh}
    %
   \begin{Karnaugh}
        \contingut{x,x,0,\textbf{x},x,x,1,\textbf{x},x,x,\textbf{x},\textbf{x},x,x,\textbf{x},\textbf{x}}
    \implicant{4}{14}{green}
    \end{Karnaugh}
\end{center}
\subsubsection{Diagrama}

\begin{center}
\begin{tabular}{|c|c|}
\hline 
$J_i$ & $K_i$ \\ 
\hline 
$J_0=1$ & $K_0=1$ \\ 
$J_1=Q_0 \overline{Q}_3$ & $K_1=Q_0$ \\ 
$J_2=Q_0 Q_1$ & $K_2=Q_0 Q_1$ \\ 
$J_3=Q_0 Q_1 Q_2$ & $K_3=Q_0$ \\ 
\hline
\end{tabular} 
\includegraphics[scale=0.3]{CircuitoLLogico}
\end{center}
Como el circuito lógico es un contador \texttt{0-9} simplemente se procede a obtener del pin que de el pulso de subida cuando llega a 9 a otro del mismo tipo, solo que este va directamente donde iría el pulso de reloj.
\section{Diseño en Proteus 8.6 $^\text{{\small \textcopyright}}$ }
\begin{center}
\includegraphics[scale=0.45]{Proteus-Circuito}
\end{center}
Para el contador \texttt{0-99} hemos realizado el contador \texttt{0-9}, este, a travez del pulso de reloj cuenta, y una vez llega a 9 se reinicia en cero. Como este princio puede ser generalizado podemos conectar un circuito bajo el mismo esquema del \texttt{7447} y \texttt{7490}, mandar desde la conección \texttt{11-6}, al pin 14 del \texttt{7490} para las decenas, y así sucesivamente para 999,9999,99999, etc...
\begin{figure}[!h]
\centering
\includegraphics[scale=0.08]{Circuito-Real2}
\caption{Foto del Circuito Final}
\end{figure}
\pagebreak
\section*{Conclusión}
En si pudimos apreciar el funcionamiento lógico de los \textit{circuitos integrados} en la manera de cómo va la información según a los pulsos de voltaje y como se puede realizar una extensión para realizar un contador hasta 999, 9999, etc. Porque los circuitos integrados al igual que la informática se basa en la lectura y escritura de números binarios.
\end{document}