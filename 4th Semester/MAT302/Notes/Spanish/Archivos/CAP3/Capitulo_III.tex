\chapter{Muestreo y Distribuciones de Muestreo}
\section{Distribuciones de Muestreo}
\subsection{Estadígrafos y Estadísticos}
\subsubsection*{Estadígrafos}
Es todo número $\widehat{\Theta}$ obtenido a partir de los datos muestrales con el proposito de estimar los parámetros poblacionales.
$$ \widehat{\Theta}_1 = f(x_1,x_2,\ldots,x_n)$$
donde $x_1,x_2,\ldots,x_n$ son los valores muestrales de las correspondientes v.a. $X_1,X_2,\ldots,X_n$, es decir un estadigrafo es un número obtenido por una función únicamente de las v.a.
$$X_1,X_2,\ldots,X_n$$
\subsubsection*{Estadística o Estadístico}
Si seleccionamos $k$ muestras aleatorias de la misma población, obtendremos $k$ valores para $\widehat{\Theta}$, tambien aleatorios. Así $\widehat{\Theta}$ es a su vez una v.a. llamada estadística o estadístico cuyos valores son estadígrafos.
\subsection{Desigualdad Chebyshev}
Si $X$ es una v.a. con una media $\mu$ y varianza $\sigma^2$ entonces:
$$ P(|x-\mu| < k\sigma ) \geq 1 - \dfrac{1}{k^2} \text{;\hspace{1cm} donde $k>0$}$$
Si $X$ es v.a. normal, entonces:
\begin{align*}
P(|x-\mu| < k\sigma ) = & P( -k\sigma < x-\mu < k\sigma ) \\
			    =  &   P( -k < z < k ) \\
			     = & \O(k)- \O(-k) \\
			      =&\O(k) - (1-\O(k))\\
			      =& 2\O(k) -1
\end{align*}

$$\therefore P(|x-\mu| < k\sigma ) = 2\O(k)-1$$
\section{Distribución de la Media Muestral}
Si:
$$X \sim f(x;\mu,\sigma^2)$$

donde: $E(x) = \mu$ y $V(x) =\sigma^2$, entonces:
$$E(\overline{x}) = \mu_{\overline{x}} = \mu$$

$$
V(\overline{x}) = \sigma^2_{\overline{x}}= 
\begin{cases}
\dfrac{\sigma^2}{n} ;&\text{si la poblacion es infinita}\\
& \\
\dfrac{\sigma^2}{n}\left(\dfrac{N-n}{N-1}\right) ;& \text{si la poblacion es finita de tamaño $N$.}
\end{cases}
$$
donde: $\dfrac{N-n}{N-1}$ es el factor de corrección por poblacion finita.
\subsubsection{Demostración}
Sea $\{x_1,x_2,\ldots,x_n\}$ donde $x_1,x_2,\ldots,x_n$ son los valores muestrales de las v.a.'s independientes tales que:

$$ \forall i\in I
\begin{cases}
E(x_i) =&\mu \\
V(x_i) =&\sigma^2
\end{cases}
$$

\begin{align*}
E(\overline{x}) =& E\left(\dfrac{\displaystyle\sum_{i=1}^{n} x_i}{n}\right) = \dfrac{1}{n}E\left(  \displaystyle\sum_{i=1}^{n} x_i \right) \\
=& \dfrac{1}{n} \left(  \displaystyle\sum_{i=1}^{n} E(x_i) \right) \\ 
=& \dfrac{1}{n}\displaystyle\sum_{i=1}^{n} \mu \\
=& \dfrac{1}{n} (n\mu) \\
=& \mu
\end{align*}
$\therefore E(\overline{x}) = \mu$
%TODO Falta hacer la demostracion de V(x)
\subsection{Distribución de Muestras de la diferencia o suma de Medias}
\subsection{Distribución de la Proporción Muestral}
\begin{enumerate}[label=\textbf{(\Roman*)}]
\item Si la estadística $\widehat{p}$ es la proporción de éxitos en una m.a. de tamaño $n$ extraída de una población binomial infinita con proporción de éxitos $p$, entonces:

 $$z=\dfrac{\widehat{p}-p}{\sqrt{\dfrac{p(1-p)}{n}}}$$
 es una v.a. cuya distribución se aproxima a la normal estándar cuando $n\rightarrow\infty$. \\
 Si la población binomial es finita de tamaño $N$, entonces:
 $$z=\dfrac{\widehat{p}-p}{\sqrt{\dfrac{p(1-p)}{n}}\left(\dfrac{N-n}{N-1}\right)}$$
 Ademas, como se pasa de una v.a. discreta (binomial) a una continua (normal), se debe introducir el factor de corrección de cantidad $\dfrac{1}{2n}$, sumando este factor al límite superior y restando al límite inferior.
 $$\hat{p}=\dfrac{x}{n}$$
Si $n$ es suficientemente grande, se puede obviar el factor de corrección:
  $$E(\hat{p})=\mu_{\hat{p}}=p$$
  $$V(\hat{p})=\sigma_{\hat{p}}^2=\dfrac{p(1-p)}{n}$$
\item Si las estadísticas $\widehat{p_1},\widehat{p_2}$ son las proporciones de éxito en m.a.'s $n_1,n_2$ extraídas de dos poblaciones binomiales infinitas con proporción de éxito $p_1,p_2$ respectivamente, entonces:
$$z=\dfrac{(\hat{p_1}-\hat{p_2})-(p_1-p_2)}{\sqrt{\dfrac{p_1(1-p_1)}{n_1}+\dfrac{p_2(1-p_2)}{n_2}}}$$
es una v.a. cuya distribución se aproxima a la normal estandar cuando $n_1,n_2\rightarrow\infty$. \\
Si las poblaciones binomiales son finitas de tamaño $N_1,N_2$ respectivamente, entonces:
$$z=\dfrac{(\hat{p_1}-\hat{p_2})-(p_1-p_2)}{\sqrt{\dfrac{p_1(1-p_1)}{n_1}\left( \dfrac{N_1-n_1}{N_1-1}\right) +\dfrac{p_2(1-p_2)}{n_2}\left( \dfrac{N_2-n_2}{N_2-1}\right) }}$$
\end{enumerate}