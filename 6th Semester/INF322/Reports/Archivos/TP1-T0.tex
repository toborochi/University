\begin{enumerate}
\item Después de revisar el vídeo explicativo de la Base de Datos \texttt{demo}, se debe responder las siguientes preguntas:
\begin{enumerate}
\item ¿Cuantas tablas tiene la Base de Datos \texttt{demo}? \\
La BD \texttt{demo} contiene 4 Tablas. \\
\item Liste los nombre de las tabla de la Base de Datos \texttt{demo} y describa que se almacena en cada tabla. \\
Las tablas de esta BD son:
\begin{itemize}
\item \textbf{Producto} (\texttt{prod}): Esta tabla almacena un código de producto, el nombre del mismo y su color.
\item \textbf{Almacén} (\texttt{alma}): Esta tabla tiene un código de almacen, el nombre del mismo y la ciudad donde se encuentra.
\item \textbf{Proveedor} (\texttt{prov}): Esta tabla se refiere a los proveedores de productos, y contiene el codigo del proveedor, el nombre del proveedor y la ciudad del mismo.
\item \textbf{Suministro} (\texttt{sumi}): Esta tabla contiene los códigos de productos, almacen y proveedor, con la fecha, la cantidad, el precio y el importe.\\{}
\end{itemize} 

\item Explique el esquema de cada una de las tablas de la BD \texttt{demo}.

\begin{itemize}
\item Tabla \texttt{prov}
\begin{itemize}
\item crpv: Código que Identifica a un proveedor.
\item nomb: Nombre del Proveedor.
\item ciud: Ciudad de donde proviene.
\end{itemize}
\item Tabla \texttt{alma}
\begin{itemize}
\item calm: Código que Identifica a un almacén.
\item noma: Nombre del Almacén,
\item ciud: Ciudad donde se encuentra el almacén.
\end{itemize}
\item Tabla \texttt{prod}
\begin{itemize}
\item cprd: Código que Identifica a un Producto.
\item nomp: Nombre del Producto.
\item colo: Color del Producto.
\end{itemize}
\item Tabla \texttt{sumi}
\begin{itemize}
\item crpv: Código que hace referencia al proveedor.
\item calm: Código que hace referencia al almacén.
\item cprd: Código que hace referencia al producto.
\item ftra: Fecha de recepcion.
\item cant: Cantidad del suministro.
\item prec: Precio del suministro.
\item impt: cant*prec (cantidad $\times$ precio).

\end{itemize}
\end{itemize}

\item ¿ Cuales son las llaves primarias y foráneas de las tablas  de la BD \texttt{demo}?

Las llaves primarias para cada tabla son:
\begin{itemize}
\item	Producto(prod): cprod
\item	Almacen(alma): calm
\item	Proveedor(prov): cprv
\end{itemize}
En este caso la unica clase que tiene llaves foráneas es:
	Suministro, ya que contiene los atributos comunes hacia las otras tablas que son sus llaves primarias.
\\
\item ¿ Cual es el campo común entre las tablas \texttt{prov} y \texttt{sumi}? \\
El campo común es el Código del Proveedor (cprv). \\

\item ¿ Cual es el campo común entre las tablas \texttt{sumi} y \texttt{prod}?

El campo común es el Código del Producto (cprd). \\
\item ¿ Cual es el campo común entre las tablas \texttt{alma} y \texttt{sumi}?

El campo común es el Código del Almacen (calm). \\
\item ¿ Cual es el campo común entre las tablas \texttt{alma} y \texttt{prov}?

Estas tablas no tienen campo en común. \\
\item Crear en SQL Server el esquema fisico de la Base de Datos \texttt{demo}
\lstinputlisting[language=SQL]{TP01.sql}
\end{enumerate}
\end{enumerate}