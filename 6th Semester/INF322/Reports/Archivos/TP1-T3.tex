\begin{enumerate}
\item ¿Cuál es el concepto de Base de Datos Consistente?

\item Al ejecutar una o más instrucciones SQL ¿Cuáles son las posibles razones para que las instrucciones SQL NO terminen de ejecutarse?
\item Al ejecutar dos o más instrucciones SQL. Defina el concepto de Ejecución Serializable.
\item Al ejecutar dos o más instrucciones SQL. Defina el concepto de Atomicidad.
\item ¿Qué es una Transacción?¨
\item ¿Las Transacciones resuelven el problema de Seriabilidad y Atomicidad?
\item ¿Cuándo usar Transacciones?
\item ¿Cuáles son los beneficios de usar Transacciones?
\item ¿Cuáles es el Rol de las Transacciones en una Base de Datos ?
\item ¿Cómo se inicia una transacción?
\item ¿Defina el concepto de COMMIT en las transacciones?
\item ¿Defina el concepto de ROLLBACK en las transacciones?
\item ¿Describa las propiedades ACID de la transacciones?
\item ¿Describa los diferentes estado de las Transacciones?
\item ¿Describa el concepto de BITACORA en una Base de Datos?
\item ¿Cuál es el rol de la BITACORA en una Base de Datos?
Escriba las instrucciones básicas que debe incluir una Transacción. 
\begin{lstlisting}[language=SQL]
DECLARE @d float
DECLARE @n float
DECLARE @r float

SET @d = 0
SET @n = 255

BEGIN TRY
	SET @r = @n/@d
END TRY
BEGIN CATCH
	PRINT 'Division entre 0 Imposible!'
END CATCH
\end{lstlisting}
\end{enumerate}
