\begin{enumerate}
\item Desarrolle las siguientes actividades
\begin{enumerate}
\item Explique por qué el Lenguaje \texttt{SQL} es un lenguaje no procedimental. \\
Se le dice no procedimental ya que nosotros hacemos consultas sobre que datos queremos, no sobre como recobrarlos.\\
\item Explique para qué sirve la clausula \texttt{SELECT} en una consulta.\\
En una consulta esta clausula nos permite especificar que datos voy a seleccionar.\\
\item Explique para qué sirve la clausula \texttt{FROM} en una consulta.\\
Esta clausula especifica de donde (las tablas) se obtienen los datos que me interesan obtener.\\
\item Explique para qué sirve la clausula \texttt{WHERE} en una consulta.\\
Esta clausula nos permite, junto con las de \texttt{SELECT} y \texttt{FROM} especificar las tuplas (o filas) que vamos a consultar y puede ir acompañada de conectores lógicos y/o relacionales.\\


\item Explique para qué sirve la clausula \texttt{GROUP BY} en una consulta.

Esta clausula nos permite agrupar los datos por atributos.
\\
\item Explique para qué sirve la clausula \texttt{HAVING} en una consulta.

Esta clausula va junto con la \texttt{GROUP BY} y sirve para aplicar cierto filtro en una consulta.\\

\item Explique para qué sirve la clausula \texttt{ORDER BY} en una consulta.

Se utiliza para  ordenar los datos obtenidos en forma ascendente o descendente de acuerdo con una o más columnas.\\

\item Es posible escribir una consulta solamente con la clausula \texttt{SELECT}

Falso \\
\item La clausula \texttt{HAVING} solamente se usa con la clausula \texttt{SELECT}

Falso \\
\item Explique qué es una sub consulta.

Es una consulta \texttt{SELECT} dentro de otra. A esta ultima se le llamaria consulta principal.\\
\item Explique como funciona la clausula \texttt{IN}

Al ser un operador le permite especificar múltiples valores en una cláusula \texttt{WHERE}. Lo utilizamos para hacer en una subconsulta para que la consulta utilice estos datos.
\\
\item Explique como funciona la clausula \texttt{EXISTS}

Al ser un operador que se utiliza para probar la existencia de cualquier registro en una subconsulta. Funciona revisando cada fila de una tabla donde hacemos una referencia externa y nos devuelve el resultado TRUE siempre que algun criterio se cumpla sino FALSE.
\\
\item Explique qué es una referencia externa.

Es cuando comparamos un campo de una subconsulta con su consulta padre.\\
\item Explique para qué sirve cada una de la siguientes funciones agregadas del SQL:
\begin{itemize}
\item \texttt{AVG()}: función que devuelve el valor promedio de una columna numérica.
\item \texttt{SUM()}: función que devuelve la suma total de una columna numérica.
\item \texttt{COUNT()}: función que devuelve el número de filas que coinciden con un criterio especificado
\item \texttt{COUNT(*)}: cuenta todas las filas en la tabla de destino, ya sea que incluyan o no NULL.
\item \texttt{MIN()}: función que devuelve el valor más pequeño de la columna seleccionada.
\item \texttt{MAX()}: función que devuelve el valor más grande de la columna seleccionada.
\end{itemize}
\end{enumerate}

\end{enumerate}