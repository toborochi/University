\begin{enumerate}
\item ¿Qué es un Trigger y para qué sirve?

Es un Procedimiento Almacenado asociado con una tabla, el cual se ejecuta automaticamente cuando se ejecuta instrucciones que modifica un dato de esa tabla.

\item ¿Cuáles son los beneficios de usar Trigger?

Mejor para las reglas complejas del negocio que no se pueden expresar como restricciones referenciales tales como actualizaciones o borrados en cascada

\item ¿Donde se almacenan los Trigger?

Los Trigger son almacenados para cada tabla asociada, en el SQL SERVER los trigger se los puede visualizar en el Explorador de Objetos dentro de la tabla.

\item ¿Para qué tipo de instrucciones se pueden definir un Trigger?

Se puede definir para las instrucciones \texttt{INSERT,UPDATE} y \texttt{DELETE}

\item Indique los casos en que se deben usar  los Trigger's
\begin{itemize}
\item Hacer modificaciones en cascada sobre tablas relacionadas
\item Deshacer cambios que violan la integridad de los datos
\item Forzar restricciones que son muy complejas para reglas y restricciones
\item Mantener datos duplicados
\item Mantener columnas con datos derivados
\item Hacer ajustes de registros
\end{itemize}

\item ¿Cuándo se activa un Trigger?

\item ¿Cual es la funcion de la tabla \texttt{INSERTED}?

Almacena cualquier fila que se vaya a añadir a la tabla

\item ¿Cual es la funcion de la tabla \texttt{DELETED}?

Almacena cualquier fila que se vaya a borrar de la tabla

\item ¿Cómo crear un Trigger?

\begin{lstlisting}[language=SQL]
CREATE TRIGGER nombre_trigger
ON nombre_tabla
AFTER  {[INSERT],[UPDATE],[DELETE]}
AS
{sql_statements}
\end{lstlisting}

\item ¿Cómo borrar un Trigger?

\begin{lstlisting}[language=SQL]
DROP TRIGGER nombre_trigger
\end{lstlisting}

\item ¿Cómo ejecutar un Trigger?

No pueden ser invocados directamente; al intentar modificar los datos de una tabla para la que se ha definido un disparador, el disparador se ejecuta automáticamente.


\item ¿Cómo editar un Trigger?

No se puede modificar un disparador, primero habria que hacerle \texttt{DROP}.

\item Explique las diferencias entre Restricciones y Trigger en una Base de Datos.

Las \textbf{restricciones} se comprueban ANTES de la ejecución de una instrucción "insert", "update" o "delete". 
Las restricciones se comprueban primero, si se infringe alguna restricción, el desencadenador no llega a ejecutarse.

Los \textbf{triggers} se ejecutan DESPUES de la ejecución de una instrucción "insert", "update" o "delete" en la tabla en la que fueron definidos.
Los triggers puden hacer referencia a los atributos de otras tablas.



\end{enumerate}
