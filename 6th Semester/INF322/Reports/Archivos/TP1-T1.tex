\begin{enumerate}
\item Como crear una función:

\begin{lstlisting}[language=SQL]
CREATE FUNCTION esquema.nombreFuncion (@nombre1 tipo1,...)
RETURNS tipo
AS
BEGIN
/* 	Cuerpo de la Funcion */
	RETURN dato
END
\end{lstlisting}
\item Como borrar una función:

\begin{lstlisting}[language=SQL]
DROP FUNCTION esquema.nombreFuncion
\end{lstlisting}

\item Como editar una función existente:

\begin{lstlisting}[language=SQL]
ALTER FUNCTION esquema.nombreFuncion (@nombre1 tipo1,...)
RETURNS tipo
AS
BEGIN
/* 	Cuerpo de la Funcion */
	RETURN dato
END
\end{lstlisting}

\item Como compilar una función:

Una vez utilizada queda precompilada en la base de datos. Es decir, se guardan en ellas. La primera vez que se ejecuta se almacena.

\item ¿Que tipos de funciones se pueden crear?

Funciones Escalares y Funciones con valores de tabla, retorna un conjunto de valores (filas y columnas).

\item ¿Como recibir y devolver parámetros en una función?

Para recibir datos en una funcion simplemente en la parte de argumentos se pasa como en una funcion cualquiera. Para devolver se puede emplear la palabra clave \texttt{exec} or \texttt{execute} seguido de la variable a la que se quiere asignar el valor. Ejemplo:

\begin{lstlisting}[language=SQL]
DECLARE @x int
DECLARE @a int = 5
DECLARE @b int = 15
EXEC @x = dbo.suma(@a,@b)
\end{lstlisting}

\item ¿Que tipos de datos se pueden crear dentro de una función?

Se pueden usar los datos que hemos usado hasta ahora.

\item ¿Que tipo de instrucción se pueden ejecutar en una función?

Se pueden realizar las instrucciones que ya hemos usando previamente, \texttt{SELECT,FROM,WHERE},etc...

\item ¿Como crear un Procedimiento Almacenado?

\begin{lstlisting}[language=SQL]
CREATE PROCEDURE nombrePA (@nombre1 tipo1,...,@nombre1 tipo1 out)
AS
BEGIN
/* 	Cuerpo del PA*/
	RETURN
END
\end{lstlisting}

\item ¿Como borrar un Procedimiento Almacenado?

\begin{lstlisting}[language=SQL]
DROP PROCEDURE nombrePA;  
GO  
\end{lstlisting}

\item ¿Como editar un Procedimiento Almacenado existente?

\begin{lstlisting}[language=SQL]
ALTER PROCEDURE nombrePA (@nombre1 tipo1,...,@nombre1 tipo1 out)
AS
BEGIN
/* 	Cuerpo del PA*/
	RETURN
END
\end{lstlisting}

\item ¿Como compilar un Procedimiento Almacenado?

De igual manera que las funciones, la primera vez que se ejecutan los comandos, estos son precompilados y almacenados.

\item ¿Como recibir y devolver parámetros en un Procedimiento Almacenado?

Puede recibir datos de la misma manera que las funciones, no devuelve parametros, lo que si hace es tener \texttt{out} al final de la declaracion de un argumento, con esto logramos que el dato de argumento sea devuelto luego de realizarle modificaciones en el PA.

\item ¿Que tipos de datos se pueden crear dentro de un Procedimiento Almacenado
?

Los mismos que las funciones.

\item ¿Que tipo de instrucción se pueden ejecutar en un Procedimiento Almacenado?

Los mismos que en funciones.

\end{enumerate}

