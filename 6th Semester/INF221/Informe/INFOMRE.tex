\documentclass[10pt,letterpaper]{report}
\usepackage[utf8]{inputenc}
\usepackage[spanish]{babel}
\usepackage{amsmath}
\usepackage{amsfonts}
\usepackage{amssymb}
\usepackage{graphicx}
\usepackage[left=2cm,right=2cm,top=2cm,bottom=2cm]{geometry}
\usepackage{xparse}

\newsavebox{\fminipagebox}
\NewDocumentEnvironment{fminipage}{m O{\fboxsep}}
 {\par\kern#2\noindent\begin{lrbox}{\fminipagebox}
  \begin{minipage}{#1}\ignorespaces}
 {\end{minipage}\end{lrbox}%
  \makebox[#1]{%
    \kern\dimexpr-\fboxsep-\fboxrule\relax
    \fbox{\usebox{\fminipagebox}}%
    \kern\dimexpr-\fboxsep-\fboxrule\relax
  }\par\kern#2
 }


\author{Leonardo H. Añez Vladimirovna}
\title{Proyecto IO2\\Analisis sobre colas en el sistema Bancario}
\begin{document}
\maketitle
\textbf{Datos Documento}
\begin{fminipage}{\textwidth}
Ubicación: Banco Economico, Calle Ingavi.\\
Cantidad: 516 Muestras
\end{fminipage}

\section*{Introducción}
Este informe pretende hallar datos respecto a el sistema Bancario, específicamente en el empleo de teoría de Colas en el estudio de las mismas. Usando una basta cantidad de datos. Se pretende dar un vistazo y analisis sobre los resultados recopilados en funcion a datos reales de la fecha \textit{18 de Noviembre}.
\section*{Estructura de los Datos}
Los datos han sido recopilados a partir de una muestra obtenida de una maquina expendedora de tickets, en una jornada completa y en una única sucursal, donde todos los clientes han sido atendidos. Se cuenta con varias cajas (es decir varias filas virtuales) ademas de distintos tipos de servicios otorgados a ciertos grupos de clientes.
\subsection*{Datos}
\begin{itemize}
\item \textbf{Cajas:} Se cuenta con hasta 10 cajas de atención.
\item \textbf{Clientela:} existen clasificaciones de servicios. En la siguiente tabla se ven reflejados.
\begin{center}
\begin{tabular}{|c|c|}
\hline 
\texttt{CN} & Caja Normal \\ 
\hline 
\texttt{CT} & Caja Tercera Edad \\ 
\hline 
\texttt{CF} & Caja Fraccionamiento \\ 
\hline 
\texttt{CE} & Caja Discapacitados \\ 
\hline 
\end{tabular} 
\end{center}
\end{itemize}
\section*{Resultados}
Los resultados a observar serán:
\begin{itemize}
\item $\bar{x}$: tiempo promedio de espera. 
\item $\bar{t}$: tiempo promedio en el sistema. 
\item $\bar{s}$: tiempo promedio en el servidor. 
\end{itemize}

\end{document}