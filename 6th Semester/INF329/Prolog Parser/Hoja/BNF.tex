\documentclass[10pt,letterpaper]{article}
\usepackage[utf8]{inputenc}
\usepackage[spanish]{babel}
\usepackage{amsmath}
\usepackage{amsfonts}
\usepackage{amssymb}
\usepackage{graphicx}

\usepackage{xparse}

\newsavebox{\fminipagebox}
\NewDocumentEnvironment{fminipage}{m O{\fboxsep}}
 {\par\kern#2\noindent\begin{lrbox}{\fminipagebox}
  \begin{minipage}{#1}\ignorespaces}
 {\end{minipage}\end{lrbox}%
  \makebox[#1]{%
    \kern\dimexpr-\fboxsep-\fboxrule\relax
    \fbox{\usebox{\fminipagebox}}%
    \kern\dimexpr-\fboxsep-\fboxrule\relax
  }\par\kern#2
 }

\usepackage{pifont}% http://ctan.org/pkg/pifont
\newcommand{\cmark}{\ding{51}}%
\newcommand{\xmark}{\ding{55}}%
\usepackage[left=2cm,right=2cm,top=2cm,bottom=2cm]{geometry}
\author{Leonardo H. Añez Vladimirovna}
\title{{\normalsize \texttt{COMPILADORES \\ \vspace{-0.5cm} INF329(SA)}}\\ Prolog List BNF}

\date{19 de Diciembre de 2019}

\begin{document}
\maketitle

Definir una $BNF$ para listas de \texttt{Prolog} cuyo elementos son \texttt{NUM}$\geq 0$. Tambien, los elementos pueden ser sumas (y restas) de estos \texttt{NUM}s. Ejemplos:
\begin{itemize}
\item \texttt{[]} \cmark
\item \texttt{[2,-4,8]} \xmark \texttt{ // Tiene un numero negativo: -4}
\item \texttt{[2+3,6-5-2+0,4,[],[27,9],9] } \cmark
\end{itemize}

Definir un \textbf{Parser} que verifique la sintaxis y también evalue la lista según este criterio:
$$
L = [e_1,e_2,e_3]
$$
\noindent Evaluación = $e_1 + e_2 + e_3$, Si $e_i$ es una lista se le resta su evaluación.
\\ ${ }$ \\
La \textbf{BNF}:
\begin{fminipage}{\textwidth}
\begin{align*}
\vspace{10cm} \\ { }  \\ { } \\ { } \\ { } \\ { } \\ { }\\ { } \\ { } \\
\end{align*}
\end{fminipage}



\end{document}