\chapter{Conceptos Generales}
Se dice que una ecuación es Diferencial si contiene una función desconocida y una o mas de sus derivadas. Si las ecuaciones contienen derivadas de funciones que dependen de una sola variable independiente se tiene una \textit{Ecuación Diferencial Ordinaria.} Si la función depende de varias variables independientes se tienen \textit{Ecuaciones Diferenciales en Derivadas Parciales.}

\subsubsection{Teorema}
Si se tiene la Ecuación:
$$F(x,y,y',y'',\ldots,y^{(n)})=0$$
Si se logra conseguir una función $\alpha(x)$ tal que, al reemplazar en la Ecuación:
$$F(x,\alpha(x),\alpha'(x),\alpha''(x),\ldots,\alpha^{(n)}(x))=0$$
Entonces se dice que $\alpha(x)$ es \textbf{Solución} de la Ecuación Diferencial.
\subsection*{Orden, Grado y Linealidad de una Ecuación Diferencial}
\begin{itemize}
\item \textbf{Orden:} El Orden de una ED\footnote{ED = Ecuaciones Diferenciales (notación que usaremos de aquí en adelante)} es el orden de la derivada mas alta de la función desconocida (variable dependiente) que aparece en la ecuación.
\item \textbf{Grado:} El grado se expresa mediante el mayor exponente de la derivada de mayor orden.
\item \textbf{Linealidad:} Una ED Lineal de orden $n$ es una ecuación de la forma:
\end{itemize}