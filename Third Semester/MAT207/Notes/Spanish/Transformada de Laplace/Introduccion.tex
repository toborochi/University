% Partes Incluidas en este Archivo
% Portada
% Notas del Autor
% Tabla de Contenidos


% Caratula 
\title{Transformada de Laplace}

\author{Leonardo H. Añez Vladimirovna}
\affil{Universidad Autónoma Gabriél René Moreno,\\Facultad de Ingeniería en Ciencias de la Computación y Telecomunicaciones, \\Santa Cruz de la Sierra, Bolivia}
\date{\today}

\maketitle

Estos apuntes fueron realizados durante mis clases en la materia \texttt{MAT207 (Ecuaciones Diferenciales)}, , en el período \texttt{I-2018} en la Facultad de Ingeniería en Ciencias de la Computación y Telecomunicaciones. 
\\ \vspace{0.5cm} \\
Para cualquier cambio, observación y/o sugerencia pueden enviarme un mensaje al siguiente correo:
\begin{center}
 \texttt{toborochi98@outlook.com}
\end{center}

\subsubsection*{Definición} Sea $F(t)$ definida para $t>0$. Entonces la Integral:

$$\mathscr{L} \lbrace F(t)\rbrace = f(s) \displaystyle\int_{0}^{\infty} e^{-st}\cdot F(t) dt$$
Es la \textit{Transformada de Laplace} siempre que la Integral Converja.