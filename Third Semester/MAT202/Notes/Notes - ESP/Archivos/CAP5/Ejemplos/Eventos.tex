\section{Ejemplos}
\begin{enumerate}
\item Tres señoritas: María, Marta y Maritza. Compiten en un concurso de belleza. Los premios son solamente otorgados a las que ocupan el primer y segundo lugar.
\begin{enumerate}
\item Listar los elementos del espacio muestral, correspondiente a los experimentos \textit{''Elegir dos Ganadores''}
\item Definir como subconjunto de $S$, los eventos:
\begin{itemize}
\item $A$: María gana el concurso de Belleza.
\item $B$: María gana el segundo lugar.
\item $C$: Maritza y Marta ganan los premios.
\end{itemize}
\end{enumerate}
\subsubsection{Solución}
Para el inciso $a$ el espacio muestral es todos los resultados en los que María gana , es decir, primer o segundo lugar. Esto es:
Para el inciso $b$ las respuestas serían:
\begin{itemize}
\item $A=\left\lbrace (Maria,Marta),(Maria,Maritza),(Marta,Maria),(Maritza,Maria)\right\rbrace $
\item $B=\left\lbrace (Marta,Maria),(Maritza,Maria)\right\rbrace$
\end{itemize}

\end{enumerate}