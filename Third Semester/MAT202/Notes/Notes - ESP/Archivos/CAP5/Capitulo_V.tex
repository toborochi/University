\chapter{Probabilidad}
Es una disciplina abstracta que se usa como modelo para hacer deducciones acerca de eventos que posiblemente puedan suceder, por tanto la probabilidad no proporciona la base para construir medidas exactas de incertidumbre, para su estudio es necesario conocer los siguientes conceptos:
\subsection*{Experimentos}
\subsubsection{Experimentos Determinísticos}
Un experimento es deterministico cuando el resultado de la observación es determinado en forma precisa, bajo las condiciones por las cuales se realiza dicho experimento. Algunos ejemplos son:
\begin{itemize}
\item Observar el color de una ficha extraída de una urna que solo contiene fichas rojas.
\item Observar el resultado de la suma de dos números naturales impares.
\end{itemize}
\subsubsection{Experimentos no Determinísticos}

\subfile{Archivos/CAP5/Secciones/Eventos}
\subfile{Archivos/CAP5/Ejemplos/Eventos}

\subfile{Archivos/CAP5/Secciones/SigmaAlgebra}
\subfile{Archivos/CAP5/Secciones/AxiomaProbabilidad}