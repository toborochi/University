\section{Cuantiles}
Como consecuencia del estudio de la mediana, surgen otros estadígrafos que dividen a las observaciones en otras proporciones y no solo en mitades como lo hace la media, los estadígrafos mas frecuentes de este tipo en el análisis estadístico son:
\begin{itemize}
\item \textbf{Cuartiles $(Q_i)$}
\item \textbf{Deciles $(D_i)$}
\item \textbf{Percentiles $(P_i)$}
\end{itemize}
Los cuantiles se emplean para describir el comportamiento de una población y sus resultados se expresan en tanto por ciento.
\subsection{Cuartiles $(Q_i)$}
Son valores que dividen a un conjunto de datos ordenados de forma ascendente o descendente en cuatro partes iguales:
\subsection{Deciles $(D_i)$}
Son valores que dividen a un conjunto de datos ordenados de forma ascendente o descendente en diez partes iguales:
\subsection{Percentiles $(P_i)$}
Son valores que dividen a un conjunto de datos ordenados de forma ascendente o descendente en cien partes iguales:
\subsection{Cálculo de Cuantiles}
\subsubsection{Para datos Originales}
\begin{itemize}
\item \textbf{Cuartiles:}
$$Q_i=x_{\frac{i(n+1)}{4}}$$
\item \textbf{Deciles:}
$$D_i=x_{\frac{i(n+1)}{10}}$$
\item \textbf{Percentiles:}
$$P_i=x_{\frac{i(n+1)}{100}}$$
\end{itemize}
\subsubsection{Para datos Agrupados}
\begin{itemize}
\item \textbf{Cuartiles:}
$$Q_i=y_{j-1}'+C_j\dfrac{i\frac{n}{4}-F_{j-1}}{f_j}$$
\item \textbf{Deciles:}
$$D_i=y_{j-1}'+C_j\dfrac{i\frac{n}{10}-F_{j-1}}{f_j}$$
\item \textbf{Percentiles:}
$$P_i=y_{j-1}'+C_j\dfrac{i\frac{n}{100}-F_{j-1}}{f_j}$$
\end{itemize}
\subsection{Ejemplos}