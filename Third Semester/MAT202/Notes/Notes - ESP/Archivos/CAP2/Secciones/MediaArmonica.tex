\section{Media Armónica $(M_h)$}
Es la media aritmética recíproca de los recíprocos de las variables:
\subsection{Cálculo}
\subsubsection{Para datos Originales}
$$M_h=\dfrac{1}{\dfrac{\frac{1}{x_1}+\frac{1}{x_2}+\frac{1}{x_3}+\ldots+\frac{1}{x_n} }{n}}=\dfrac{1}{\dfrac{\displaystyle\sum_{i=1}^{n}\frac{1}{x_i}}{n}}=\dfrac{n}{\displaystyle\sum_{i=1}^{n}\frac{1}{x_i}}$$
\subsubsection{Para datos Agrupados}
$$M_h=\dfrac{1}{\dfrac{\frac{f_1}{y_1}+\frac{f_2}{y_2}+\frac{f_3}{y_3}+\ldots+\frac{f_k}{y_k} }{n}}=\dfrac{1}{\dfrac{\displaystyle\sum_{i=1}^{k}\frac{f_i}{x_i}}{n}}=\dfrac{n}{\displaystyle\sum_{i=1}^{k}\frac{f_i}{y_i}}$$
\subsection{Usos de la Media Armónica}
\begin{enumerate}
\item Es apropiado para promediar velocidades.
\item En economía se usa en los índices de precio.
\item Se usa para promediar precio cuando la cantidad es variable y el monto gastado igual.
\end{enumerate}
\subsection{Ejemplos}