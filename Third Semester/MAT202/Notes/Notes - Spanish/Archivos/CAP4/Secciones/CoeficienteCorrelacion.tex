\section{Coeficiente de Correlación Rectilíneo}
\subsection{Propiedades}
\begin{enumerate}
\item $-1\leq r \leq 1$
\item Si $r>0$, entonces existe correlación directa positiva.
\item Si $r<0$, entonces existe correlación inversa negativa.
\item $r^2=0$, los datos forman una línea recta, en caso de correlación rectilínea.
\item Si $r=1$, entonces existe correlación perfecta positiva.
\item Si $r=-1$, entonces existe correlación perfecta negativa.
\item Si $r=0$, los datos son incorrelacionados.
\end{enumerate}
\subsubsection{Observaciones}
\begin{itemize}
\item El signo de $r$ es el mismo signo de $b$ de la ecuación de regresión: $\hat{y}=a+bx$
\item En la interpretación clásica se sostiene:
\begin{enumerate}
\item Si $0.00 \leq r < 0.20$, existe correlación no significativa.
\item Si $0.20 \leq r < 0.40$, existe baja correlación.
\item Si $0.40 \leq r < 0.70$, existe correlación significativa.
\item Si $0.70 \leq r < 1.00$, existe alto grado de asociación.
\end{enumerate}
\end{itemize}
\subsection{Cálculo}
\subsubsection{Mediante Fórmula propuesta por Pearson}
$$r=
\dfrac{n\cdot \displaystyle\sum_{i=1}^{n}x_i y_i -  \left( \displaystyle\sum_{i=1}^{n}x_i\right) \left( \displaystyle\sum_{i=1}^{n}y_i\right) }
{\sqrt{\left[ n\cdot\displaystyle\sum_{i=1}^{n}x_i^2 - \left( \displaystyle\sum_{i=1}^{n}x_i\right)^2\right]
\left[ n\cdot\displaystyle\sum_{i=1}^{n}y_i^2 - \left( \displaystyle\sum_{i=1}^{n}y_i\right)^2\right]
 }}$$

\subsubsection{En términos de Covarianza y Desviación Estándar}
$$r=\dfrac{Cov(x,y)}{S_x\cdot S_y}$$
\subsubsection{Error Estándar de Estimación}
$$S_{yx}=\sqrt{\dfrac{\displaystyle\sum_{i=1}^{n}y_i^2-a\cdot \displaystyle\sum_{i=1}^{n}y_i-b\cdot\displaystyle\sum_{i=1}^{n}y_i x_i}{n}}$$