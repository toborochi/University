\documentclass[10pt,letterpaper]{book}
\usepackage[utf8]{inputenc}
\usepackage[spanish]{babel}
\usepackage{amsmath}
\usepackage{amsfonts}
\usepackage{amssymb}
\usepackage{graphicx}
\usepackage[left=2cm,right=2cm,top=2cm,bottom=2cm]{geometry}
\author{
{\Large \textbf{Grupo Nº 7}}\\${ }$\\
\textbf{Docente:} Oscar Flores \\${ }$\\
\textbf{Integrantes:}\\
Leonardo H. Añez Vladimirovna
}
\title{
{\large \texttt{ADM100 - GRUPO SC}}\\\vspace{3cm}
{\Huge El Control}
}

\setcounter{secnumdepth}{0} % sections are level 1

\begin{document}
\maketitle
\tableofcontents

\newpage

\section{Introducción}
El control ha sido definido bajo dos grandes perspectivas, una perspectiva limitada y una perspectiva amplia. Desde la perspectiva limitada, el control se concibe como la verificación a posteriori de los resultados conseguidos en el seguimiento de los objetivos planteados y el control de gastos invertido en el proceso realizado por los niveles directivos donde la estandarización en términos cuantitativos, forma parte central de la acción de control.
\subsection{Concepto}
El control es una etapa primordial en la administración, pues, aunque una empresa cuente con magníficos planes, una estructura organizacional adecuada y una dirección eficiente, el ejecutivo no podrá verificar cuál es la situación real de la organización i no existe un mecanismo que se cerciore e informe si los hechos van de acuerdo con los objetivos.\\${ }$\\
El concepto de control es muy general y puede ser utilizado en el contexto organizacional para evaluar el desempeño general frente a un plan estratégico.
A fin de incentivar que cada uno establezca una definición propia del concepto se revisara algunos planteamientos de varios autores estudiosos del tema:
\subsection{Elementos de concepto}


\end{document}
