\documentclass[10pt,letterpaper]{book}
\usepackage[utf8]{inputenc}
\usepackage[spanish]{babel}
\usepackage{amsmath}
\usepackage{amsfonts}
\usepackage{amssymb}
\usepackage{graphicx}
\usepackage{empheq}
\usepackage{svg}
\usepackage{hyperref}
\usepackage{hypcap}
\usepackage[spanish,onelanguage,linesnumbered,ruled,vlined]{algorithm2e}
\usepackage{titling}
\pretitle{%
  \begin{flushright}
  \vspace{-9.5cm}
%  \includegraphics[width=5cm,natwidth=472,natheight=531]{logo} \\[7cm]
  \includegraphics[width=5cm]{logo} \\[6cm]
  \end{flushright}
  \begin{center}
  \LARGE
}
\posttitle{\end{center}}
\usepackage[left=4cm,right=2.5cm,top=3cm,bottom=4cm,includehead,includefoot,headheight=15pt]{geometry}
\usepackage{setspace}
\setstretch{0.99}
\usepackage{fancyhdr}
%\date{27 de Noviembre 2018}
\fancyhf{}
\renewcommand{\headrulewidth}{0pt} % optional
%\fancyhead[L]{\nouppercase{\leftmark} \hfill Section \nouppercase{\rightmark}}
\fancyhead[L]{Método Simplex}
\cfoot{\thepage}
\pagestyle{fancy}
\author{
Leonardo Henry Añez Vladimirovna\\
\texttt{Reg: 217002498}
\and
Sebastián Durán\\
\texttt{Reg: 217050662}
}
\title{
Investigación Operativa I (MAT329)\\ ${ }$\\
\textbf{Aplicación del Método Simplex en la Panadería: ''Barrio Lindo''}
\\ ${ }$\\
\small Facultad de Ingeniería en Ciencias de la Computación y Telecomunicaciones\\}

\begin{document}
\maketitle
\section*{Problema}
En este trabajo se plantea utilizar el Método Simplex en la resolución de un problema de Programación Lineal. Para realizar el analisis de producción de panes.
\section*{Información Técnica}
\section*{Modelo Matemático}
El modelo matemático tendra el siguiente esquema:
\subsection*{Variables}
Las variables del trabajo serán las cantidades por tipos de pan que se venden y estarán representados por:
$$
x_1,x_2,x_3,x_4,\ldots,x_n
$$
\subsection*{Función Objetivo}
La función objetivo estará dada de la  siguiente manera:
$$
z = Ax_1 + Bx_2 + Cx_3 + Dx_4 + Ex_5 + \cdot + Xx_n
$$
\subsection*{Restricciones}
Las restricciones al problema serán las siguientes:
\section*{Resultados}
\section*{Conclusión}
\end{document}