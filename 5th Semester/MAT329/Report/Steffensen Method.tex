\documentclass[10pt,letterpaper]{book}
\usepackage[utf8]{inputenc}
\usepackage[spanish]{babel}
\usepackage{amsmath}
\usepackage{amsfonts}
\usepackage{amssymb}
\usepackage{graphicx}
\usepackage{empheq}
\usepackage{svg}
\usepackage{hyperref}
\usepackage{hypcap}
\usepackage[spanish,onelanguage,linesnumbered,ruled,vlined]{algorithm2e}
\usepackage{titling}
\pretitle{%
  \begin{flushright}
  \vspace{-10cm}
%  \includegraphics[width=5cm,natwidth=472,natheight=531]{logo} \\[7cm]
  \includegraphics[width=5cm]{logo} \\[6cm]
  \end{flushright}
  \begin{center}
  \LARGE
}
\posttitle{\end{center}}
\usepackage[left=3cm,right=2.5cm,top=3cm,bottom=2.5cm,includehead,includefoot,headheight=15pt]{geometry}
\usepackage{setspace}
\setstretch{0.99}
\usepackage{fancyhdr}
%\date{27 de Noviembre 2018}
\fancyhf{}
\renewcommand{\headrulewidth}{0pt} % optional
%\fancyhead[L]{\nouppercase{\leftmark} \hfill Section \nouppercase{\rightmark}}
\fancyhead[L]{Método Simplex}
\cfoot{\thepage}
\pagestyle{fancy}
\author{
Leonardo Henry Añez Vladimirovna\\
\texttt{Reg: [\hspace{3cm}]}
\and
Sebastián Durán\\
\texttt{Reg: [\hspace{3cm}]}
}
\title{
Investigación Operativa I (MAT329)\\ ${ }$\\
\textbf{Aplicación del Método Simplex en la Panadería: ''Barrio Lindo''}
\\ ${ }$\\
\small Facultad de Ingeniería en Ciencias de la Computación y Telecomunicaciones\\}

\graphicspath{ {./images/} }
\begin{document}
\maketitle
\section*{Problema}
En este trabajo se plantea utilizar el Método Simplex en la resolución de un problema de Programación Lineal. Para realizar el analisis de producción de panes en la Panadería \textit{''Barrio Lindo''}. Buscando la minimización de los costos de produccion basados en un lote de una arroba para distintos tipos de panes.
\section*{Información Técnica}
Contamos con el precio unitario de los ingredientes representativos usados para preparar diferentes panes, esto para un lote.
\begin{center}
\includegraphics[scale=0.66]{Tabla}
\end{center}
Además de contar con cierta disponibilidad por cada ingrediente:
\begin{center}
\includegraphics[scale=0.66]{Tabla2}
\end{center}
\section*{Modelo Matemático}
El modelo matemático tendra el siguiente esquema:
\subsection*{Variables}
Las variables del trabajo serán las cantidades por tipos de pan que se venden y estarán representados por:
\begin{align*}
x_1,x_2,x_3,x_4,x_5,x_6
\end{align*}
\subsection*{Función Objetivo}
La función objetivo estará dada de la  siguiente manera:
$$
z = 82x_1 + 104.5x_2 + 81x_3 + 78x_4 + 80x_5 + 74x_6
$$
\subsection*{Restricciones}
Las restricciones al problema serán las siguientes:

\begin{align*}
9x_1 + 9x_2 + 10x_3 + 8x_4 + x_5 + 7x_6 &\geq 46 \\
9x_1 + 9x_2 + 10x_3 + 8x_4 + x_5 + 7x_6 &\leq 92 \\
x_1 + x_2 + x_3 + x_4 + 8x_5 + 2x_6 &\leq 20 \\
20x_1 + 20x_2 + 10x_3 + 20x_4 + 10x_5 + 10x_6 &\geq 200 \\
20x_1 + 20x_2 + 10x_3 + 20x_4 + 10x_5 + 10x_6 &\leq 400 \\
x_1 + 2x_2 + x_3 + x_4 + x_5 + x_6 &\leq 20 \\
x_1 +1.5x_2 + x_3 + x_4 + x_5 + x_6 &\leq 100 \\
x_1 + x_2 + 2x_3 + x_4 + x_5 + x_6 &\leq 46
\end{align*}
\section*{Resultados}
Mediante el uso del programa realizado para el proyecto,procedemos a cargar el programa con las especificaciones necesarias:
\begin{center}
\includegraphics[scale=0.55]{Programa1}
\end{center}
Una vez hacemos click en \texttt{Resolver} nos genera todas las iteraciones hasta llegar a la solución final. \\${ }$\\
\texttt{Iteraciones del Programa}
\begin{center}

\fbox{\includegraphics[scale=0.55]{Programa2}}
\fbox{\includegraphics[scale=0.55]{Programa3}}
\fbox{\includegraphics[scale=0.55]{Programa4}}
\fbox{\includegraphics[scale=0.55]{Programa5}}
\fbox{\includegraphics[scale=0.55]{Programa6}}
\end{center}
\pagebreak
Finalmente podemos ver en el programa el resultado final de la minimización.
\begin{center}
\includegraphics[scale=0.55]{Programa7}
\end{center}
Por lo que tenemos los resultados para la minimizacion de $z$:
\begin{align*}
Min_z &= 372.6 \\ 
x_1   &= 0  \\
x_2   &= 0  \\
x_3   &= 4.6  \\
x_4   &= 0  \\
x_5   &= 0  \\
x_6   &= 0  
\end{align*}

\section*{Conclusión}
Por los datos obtenidos podemos concluir que para las restricciones dadas, basadas en la disponibilidad, la panadería barrio lindo necesita producir únicamente el pan Libro. Para el caso de minimización de perdidas.
\end{document}