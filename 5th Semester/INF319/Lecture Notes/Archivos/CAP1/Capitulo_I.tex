\chapter{Preliminares Formales}
\section{Conjuntos}
\subsection{Conjunto Finito e Infinito}
\subsubsection{Equivalencia} Dado $A$ y $B$ (conjuntos) los llamamos \textit{equivalentes} si existe una biyección: $f:A\rightarrow B$
\subsubsection{Conjunto Finito}
Un conjunto $A$ es finito si es equivalente a $\{1,2,3,\ldots ,n\}$ para algún $n\in\mathbb{N}$.
\subsubsection{Conjunto Infinito}
Un conjunto es infinito si no es finito. Si no es equivalente a $\{1,2,3,\ldots , n\}$ es decir no hay biyección. Sin embargo no todos los conjuntos finitos son equivalentes.
\begin{itemize}
\item \textbf{Conjunto Contablemente Infinito:}  Se dice que un conjunto es contablemente infinito si es equivalente con $\mathbb{N}$.
\item \textbf{Conjunto Contable:} Es contable si es finito o contablemente infinito.
\item \textbf{Conjunto Incontable:} Se dice que es incontable si no es contable.  
\end{itemize}

\subsubsection{Principio de las Casillas}
Si $A$ y $B$ son conjuntos finitos no vacíos y $|A|>|B|$ entonces no existe una función inyectiva de: $A\rightarrow B$.
\section{Preliminares}
\subsection{Alfabeto}
Un alfabeto $\Sigma$ es cualquier conjunto finito no vacío. 
\subsubsection{Ejemplo(s)}
\begin{align*}
\Sigma_1 =& \{ Leo, Martha \} \\
\Sigma_2 =& \{0,1,2,3,\ldots , 13\} \\
\Sigma_3 =& \{ a,b \} \\
\Sigma_4 =& \{ R,G,B,A\}
\end{align*}
\subsection{Palabra}
Una palabra sobre $\Sigma$ es una sucesión finita de símbolos de $\Sigma$. Es decir:
$$(\sigma_1,\sigma_2,\ldots,\sigma_n);\sigma \in \Sigma \text{\hspace{0.5cm}  ó  \hspace{0.5cm}} \sigma_1\sigma_2\sigma_3\ldots\sigma_n ; \sigma\in\Sigma$$
\subsubsection{Ejemplo(s)}
\noindent
\begin{multicols}{4}
\noindent
\begin{align*}
\textbf{Sobre } \Sigma_1& \\
w_1 =& LeoLeo \\
w_2 =& MarthaLeoMartha
\end{align*} 
\columnbreak
\begin{align*}
\textbf{Sobre } \Sigma_2& \\
w_1 =& 1111110 \\
w_2 =& 11235813
\end{align*}
\columnbreak
\begin{align*}
\textbf{Sobre } \Sigma_3& \\
w_1 =& bababababa \\
w_2 =& abba
\end{align*}
\columnbreak
\begin{align*}
\textbf{Sobre } \Sigma_4& \\
w_1 =& ABGR \\
w_2 =& RRRA
\end{align*}
\end{multicols}
Denotamos por $\Sigma^*$ el conjunto de todas las palabras sobre $\Sigma$.
\subsubsection{Longitud de una Palabra}
Sea $w$ una palabra sobre $\Sigma$, es decir $w=\sigma_1\sigma_2\ldots\sigma_n ; \sigma\in\Sigma$. La longitud de $w$ es $n$ y se denota por: $|w|=n$.
\subsubsection{Palabra vacía}
Es la sucesión vacía de símbolos de $\Sigma$ y se denota por: $\lambda$.
\subsection{Notaciones}
\begin{itemize}
\item $\Sigma^+ = \{ w\in\Sigma^* / |w|>0\}$
\item $\Sigma^k = \{ w\in\Sigma^* / |w|=k\}$
\item $\Sigma^0 = \{ w\in\Sigma^* / |w|=0\}=\{\lambda\}$
\item $\Sigma^1 = \{ w\in\Sigma^* / |w|=1\}=\Sigma$
\item $\Sigma^* = \Sigma^+ \cup \{\lambda\}$
\item $\Sigma^+ = \Sigma^* - \{\lambda\}$
\end{itemize}
\subsubsection{Cantidad de Ocurrencias}
Sea $w\in\Sigma^*$, denotamos por $|w|_\sigma$ al número de ocurrencias del símbolo $\sigma$ en la palabra $w$.
\subsubsection{Ejemplo(s)}
$$\Sigma = \{ a,b \} $$

\begin{itemize}
\item $\Sigma^{*}= \{\lambda, a,b,aa,bb,ab,ba,aaa,\ldots\}$
\item $\Sigma^0=\{\lambda\}$
\item $\Sigma_1 =\Sigma = \{a,b\}$
\item $\Sigma^2 = \{ ab,aa,ba,bb \}$
\end{itemize}
\subsubsection{Concatenación}
Sea $u,v\in\Sigma^*$ tal que $u=\sigma_1\sigma_2\ldots\sigma_n, v=\epsilon_1\epsilon_2\ldots\epsilon_n$. La concatenación de $u$ y $v$ se define por:
$$uv=\sigma_1\sigma_2\ldots\sigma_n \epsilon_1\epsilon_2\ldots\epsilon_n$$
\subsubsection{Definición de Recurrencia}
$$|\text{ }|:\Sigma^*\rightarrow\mathbb{N}$$
$$
\begin{cases}
|\lambda |=0 \\
|wa| = |w| + 1
\end{cases}
$$
\subsubsection{Ejemplo(s)}
\begin{align*}
u=abab & uv=ababbba \\
v=bba  & vu=bbaabab
\end{align*}
\subsubsection{Definiciones}
\begin{itemize}
\item $uv\neq vu$
\item $(uv)w=u(vw)$
\item $u\lambda=\lambda u=u$
\item $|uv|=|u|+|v|$
\item $|uv|_a = |u|_a + |v|_a$
\end{itemize}
\subsection{Principio de Inducción para $\Sigma^*$}
Sea $L$ un conjunto de palabras sobre $\Sigma$ con las propiedades:
\renewcommand{\labelenumi}{\theenumi}
\renewcommand{\theenumi}{\textbf{\roman{enumi}.)}}%
\begin{enumerate}
\item $\lambda \in L$
\item $w\in L \wedge a\in\Sigma \Rightarrow wa \in L$
\end{enumerate}
\subsubsection{Entonces}
$L=\Sigma^*$, es decir, todas las palabras sobre $\Sigma$ están en $L$.