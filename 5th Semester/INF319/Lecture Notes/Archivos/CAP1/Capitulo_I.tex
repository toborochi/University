\chapter{Preliminares Formales}
\section{Conjuntos}
\subsection{Conjunto Finito e Infinito}
\subsubsection{Equivalencia} Dado $A$ y $B$ (conjuntos) los llamamos \textit{equivalentes} si existe una biyección: $f:A\rightarrow B$
\subsubsection{Conjunto Finito}
Un conjunto $A$ es finito si es equivalente a $\{1,2,3,\ldots ,n\}$ para algún $n\in\mathbb{N}$.
\subsubsection{Conjunto Infinito}
Un conjunto es infinito si no es finito. Si no es equivalente a $\{1,2,3,\ldots , n\}$ es decir no hay biyección. Sin embargo no todos los conjuntos finitos son equivalentes.
\begin{itemize}
\item \textbf{Conjunto Contablemente Infinito:}  Se dice que un conjunto es contablemente infinito si es equivalente con $\mathbb{N}$.
\item \textbf{Conjunto Contable:} Es contable si es finito o contablemente infinito.
\item \textbf{Conjunto Incontable:} Se dice que es incontable si no es contable.  
\end{itemize}

\subsubsection{Principio de las Casillas}
Si $A$ y $B$ son conjuntos finitos no vacíos y $|A|>|B|$ entonces no existe una función inyectiva de: $A\rightarrow B$.
\section{Preliminares}
\subsection{Alfabeto}
Un alfabeto $\Sigma$ es cualquier conjunto finito no vacío. 
\subsubsection{Ejemplo(s)}
\begin{align*}
\Sigma_1 =& \{ Leo, Martha \} \\
\Sigma_2 =& \{0,1,2,3,\ldots , 13\} \\
\Sigma_3 =& \{ a,b \} \\
\Sigma_4 =& \{ R,G,B,A\}
\end{align*}
\subsection{Palabra}
Una palabra sobre $\Sigma$ es una sucesión finita de símbolos de $\Sigma$. Es decir:
$$(\sigma_1,\sigma_2,\ldots,\sigma_n);\sigma \in \Sigma \text{\hspace{0.5cm}  ó  \hspace{0.5cm}} \sigma_1\sigma_2\sigma_3\ldots\sigma_n ; \sigma\in\Sigma$$
\subsubsection{Ejemplo(s)}
\noindent
\begin{multicols}{4}
\noindent
\begin{align*}
\textbf{Sobre } \Sigma_1& \\
w_1 =& LeoLeo \\
w_2 =& MarthaLeoMartha
\end{align*} 
\columnbreak
\begin{align*}
\textbf{Sobre } \Sigma_2& \\
w_1 =& 1111110 \\
w_2 =& 11235813
\end{align*}
\columnbreak
\begin{align*}
\textbf{Sobre } \Sigma_3& \\
w_1 =& bababababa \\
w_2 =& abba
\end{align*}
\columnbreak
\begin{align*}
\textbf{Sobre } \Sigma_4& \\
w_1 =& ABGR \\
w_2 =& RRRA
\end{align*}
\end{multicols}
Denotamos por $\Sigma^*$ el conjunto de todas las palabras sobre $\Sigma$.
\subsubsection{Longitud de una Palabra}
Sea $w$ una palabra sobre $\Sigma$, es decir $w=\sigma_1\sigma_2\ldots\sigma_n ; \sigma\in\Sigma$. La longitud de $w$ es $n$ y se denota por: $|w|=n$.
\subsubsection{Palabra vacía}
Es la sucesión vacía de símbolos de $\Sigma$ y se denota por: $\lambda$.
\subsection{Notaciones}
\begin{itemize}
\item $\Sigma^+ = \{ w\in\Sigma^* / |w|>0\}$
\item $\Sigma^k = \{ w\in\Sigma^* / |w|=k\}$
\item $\Sigma^0 = \{ w\in\Sigma^* / |w|=0\}=\{\lambda\}$
\item $\Sigma^1 = \{ w\in\Sigma^* / |w|=1\}=\Sigma$
\item $\Sigma^* = \Sigma^+ \cup \{\lambda\}$
\item $\Sigma^+ = \Sigma^* - \{\lambda\}$
\end{itemize}
\subsection{Cantidad de Ocurrencias}
Sea $w\in\Sigma^*$, denotamos por $|w|_\sigma$ al número de ocurrencias del símbolo $\sigma$ en la palabra $w$.
\subsubsection{Ejemplo(s)}
$$\Sigma = \{ a,b \} $$

\begin{itemize}
\item $\Sigma^{*}= \{\lambda, a,b,aa,bb,ab,ba,aaa,\ldots\}$
\item $\Sigma^0=\{\lambda\}$
\item $\Sigma_1 =\Sigma = \{a,b\}$
\item $\Sigma^2 = \{ ab,aa,ba,bb \}$
\end{itemize}
\subsection{Concatenación}
Sea $u,v\in\Sigma^*$ tal que $u=\sigma_1\sigma_2\ldots\sigma_n, v=\epsilon_1\epsilon_2\ldots\epsilon_n$. La concatenación de $u$ y $v$ se define por:
$$uv=\sigma_1\sigma_2\ldots\sigma_n \epsilon_1\epsilon_2\ldots\epsilon_n$$
\subsubsection{Definición de Recurrencia}
$$|\text{ }|:\Sigma^*\rightarrow\mathbb{N}$$
$$
\begin{cases}
|\lambda |=0 \\
|wa| = |w| + 1
\end{cases}
$$
\subsubsection{Ejemplo(s)}
\begin{align*}
u=abab & \text{ } &uv=ababbba \\
v=bba  & \text{ } & vu=bbaabab
\end{align*}
\subsubsection{Propiedades}
\begin{itemize}
\item $uv\neq vu$
\item $(uv)w=u(vw)$
\item $u\lambda=\lambda u=u$
\item $|uv|=|u|+|v|$
\item $|uv|_a = |u|_a + |v|_a$
\end{itemize}
\subsection{Inversa}
Si $w=\sigma_1,\sigma_2,\ldots ,\sigma_n \in\Sigma^n$ entonces $w'=\sigma_n,\sigma_{n-1},\ldots ,\sigma_1$ se llama inversa o transpuesta de $w$.
\subsubsection*{Definición de Recurrencia}
$$':\Sigma^*\rightarrow\Sigma^*$$
$$
\begin{cases}
\lambda'=\lambda \\
(wa)' = aw'
\end{cases}
$$
\subsection{Potencia de una Palabra}
$$w^n =  \underbrace{ww\ldots w}_{n-veces}$$
\subsubsection*{Definición de Recurrencia}
$$':\Sigma^*\rightarrow\Sigma^*$$
$$
\begin{cases}
w^0=\lambda \\
w^{n+1} = ww^{n}
\end{cases}
$$
\subsubsection{Propiedades}
\begin{itemize}
\item $|w^n| = n|w|$
\item $w^m w^n = w^{m+n}$
\item $(w^n)^m = w^{mn}$
\item $\lambda^n = \lambda$
\end{itemize}
\subsection{Principio de Inducción para $\Sigma^*$}
Sea $L$ un conjunto de palabras sobre $\Sigma$ con las propiedades:
\renewcommand{\labelenumi}{\theenumi}
\renewcommand{\theenumi}{\textbf{\roman{enumi}.)}}%
\begin{enumerate}
\item $\lambda \in L$
\item $w\in L \wedge a\in\Sigma \Rightarrow wa \in L$
\end{enumerate}
\subsubsection{Entonces}
$L=\Sigma^*$, (es decir, todas las palabras sobre $\Sigma$ están en $L$.)
\subsection{Lenguajes}
Un lenguaje sobre $\Sigma$ es un subconjunto de $\Sigma^*$
\subsubsection{Operaciones}
Recordemos que ya conocemos otras operaciones (Unión, Intersección, Diferencia y Complemento), para esta materia tenemos las siguientes:
\begin{itemize}
\item Concatenación \\ ${ }$ \\
Sea $A,B \subseteq \Sigma^*$ \\
$$AB =\{ w\in\Sigma^* / w=xy, x\in A, y\in B \}$$
\item Transposición \\ ${ }$ \\
Sea $A\subseteq \Sigma^*$ \\
$$A'=\{ w'\in \Sigma^* /w\in A\} $$
\item Estrella de Kleene \\ ${ }$ \\
Sea $A\subseteq \Sigma^*$ \\
$$A^* = \{w\in\Sigma^* / w=w_1 w_2 \ldots w_n \text{ para algún } k\in\mathbb{N} \text{ y para algunas }w_1,w_2,\ldots,w_k \in A\}$$
\end{itemize}
\subsection{Expresiones Regulares}
Las expresiones regulares (ER) sobre un alfabeto ($\Sigma$) son las palabras sobre el alfabeto $\Sigma \cup {\color{red}\{ ), (, \emptyset, \cup,*\}}$ tal que cumple lo siguiente:
\renewcommand{\labelenumi}{\theenumi}
\renewcommand{\theenumi}{\textbf{\arabic {enumi}.)}}%
\begin{enumerate}
\item {\color{red}$\emptyset$} y cada símbolo de $\Sigma$ es una ER.
\item Si $\alpha$ y $\beta$ son ER entonces {\color{red}(}$\alpha\beta${\color{red})} es una ER.
\item Si $\alpha$ y $\beta$ son ER entonces {\color{red}(}$\alpha{\color{red}\cup}\beta${\color{red})} es una ER.
\item Si $\alpha$ es una ER entonces $\alpha^{{\color{red}*}}$ es una ER.
\item Nada mas es una ER a menos que provenga de \textbf{(1.)} a \textbf{(4.)}
\end{enumerate}
\subsubsection*{Ejemplo(s)}
Para $\Sigma = \{ a,b \}$ podemos formar:
\begin{align*}
{{\color{red}\emptyset}}\\
a\\
b\\
{{\color{red}(}}ab{{\color{red})}}\\
{{\color{red}(}}a{\color{red}\cup}b{{\color{red})}}\\
{{\color{red}(}}{\color{red}(}ab{\color{red})}{\color{red}\cup}b{{\color{red})}}\\
{{\color{red}(}}ba{{\color{red})^*}}\\
{\color{red}((}ba{\color{red})^*\cup(}a{\color{red}\cup}b{\color{red}))^*}
\end{align*}
\subsubsection*{Lenguaje Regular}
Un lenguaje es regular ssi es generado por una expresión regular.
\subsection{Módulos}
\subsubsection*{Definición}
Un módulo es una tripleta $D=(k,\Sigma,f)$ donde:
\begin{itemize}
\item $k$ es un conjunto finito no vacío, \textit{llamado conjunto de estados}
\item $\Sigma$ es un conjunto finito no vacío, \textit{llamado alfabeto}
\item $f:k\times\Sigma\rightarrow k$, \textit{llamado función de transición}
\end{itemize}
\subsubsection*{Interpretación}
Un módulo se puede interpretar como un dispositivo que en determinados instantes de tiempo recibe señales (símbolos del alfabeto), que producen cambios en su configuración interna.
\begin{center}
\begin{tikzpicture}[>={Triangle[width=1.5mm,length=1.5mm]},->,node distance=2cm,auto]
\node[] (q_0) {${ }$};
\node[state,rectangle] (q_1) at (2,0) {$s\in k$};

\path[->,line width=0.25mm] (q_0) edge node[] {$\sigma\in\Sigma$} (q_1);
%\path[->] (q_0) edge node[swap] {$\sigma$} (q_1);
\end{tikzpicture}
\end{center}
\subsubsection*{Representación}
\begin{itemize}
\item \textbf{Tabla de Transición}
\item \textbf{Grafo} \\

\begin{figure}[h]
\centering
\begin{tikzpicture}[>={Triangle[width=1.5mm,length=1.5mm]},->,node distance=2cm,auto]
\node[state] (q_0) {$s_i$};
\node[state] (q_1) at (3,0) {$s_j$};

\path[->,line width=0.25mm] (q_0) edge node[] {$\sigma$} (q_1);
%\path[->] (q_0) edge node[swap] {$\sigma$} (q_1);
\end{tikzpicture}
\caption{ssi: $f(s_i,\sigma)=s_j$}
\end{figure}
\end{itemize}
\subsubsection{Comportamiento Dinámico}
Sea $D=(k,\sigma,f)$ un módulo: \\
\begin{center}
\begin{tikzpicture}[>={Triangle[width=1.5mm,length=1.5mm]},->,node distance=2cm,auto]
\node[] (s_0) {$s_0$};
\node[] (s_1) at (2,0) {$s_1$};
\node[] (s_2) at (4,0) {$s_2$};
\node[] (sus) at (6,0) {$\cdots$};
\node[] (s_k) at (8,0) {$s_k$};

\node[] (t_0) at (0,1) {$t_0$};
\node[] (t_1) at (2,1) {$t_1$};
\node[] (t_2) at (4,1) {$t_2$};
\node[] (tsus) at (6,1) {$\cdots$};
\node[] (t_k) at (8,1) {$t_k$};

\path[->,line width=0.25mm] (s_0) edge node[] {$\sigma_0$} (s_1);
\path[->,line width=0.25mm] (s_1) edge node[] {$\sigma_1$} (s_2);
\path[->,line width=0.25mm] (s_2) edge node[] {$\sigma_2$} (sus);
\path[->,line width=0.25mm] (sus) edge node[] {$\sigma_{k-1}$} (s_k);
%\path[->] (q_0) edge node[swap] {$\sigma$} (q_1);
\end{tikzpicture}
\end{center}
\subsubsection{Función Estado Terminal}
Sea $D=(k,\sigma,f)$ un módulo: \\${ }$\\
Una función de Estado Terminal del módulo $D$ es una única función:
$$
\widehat{f}:k\times\Sigma \rightarrow k \text{ tal que } \forall s\in k, w\in\Sigma^* , \sigma\in\Sigma
$$
$$
\begin{cases}
\widehat{f}(s,\lambda)=s \\
\widehat{f}(s,\sigma w)= \widehat{f}\left[ f(s,\sigma),w\right]
\end{cases}
$$
\subsubsection{$\diamond$ Notas}
\begin{itemize}
\item $w=\lambda$
$$
{\color{red}\widehat{f}(s,\sigma)}=\widehat{f}(s,\sigma\lambda)=\widehat{f}\left[ f(s,\sigma),\lambda\right]={\color{red}f(s,\sigma)}
$$
\item $\forall w\in\Sigma^*$
$$
f:k \rightarrow k \text{ tal que: } f_w(s)=\widehat{f}(s,w)
s
$$
\end{itemize}

\subsection{Máquinas}
Una máquina es una quíntupla $M=(k,\Sigma,\Delta,f,g)$ donde:
\begin{itemize}
\item $k$ es un conjunto finito no vacío, \textit{llamado conjunto de estados}
\item $\Sigma$ es un conjunto finito no vacío, \textit{llamado alfabeto de entrada}
\item $\Delta$ es un conjunto finito no vacío, \textit{llamado alfabeto de salida}
\item $f:k\times\Sigma\rightarrow k$, \textit{llamado función de transición}
\item $g:k\times\Sigma\rightarrow \Delta$, \textit{llamado función de salida}
\end{itemize}
\subsubsection{Interpretación}
Una máquina se puede interpretar como un dispositivo que en determinados instantes de tiempo recibe señales (símbolos de entrada) que producen cambios en su configuración interna y emiten señales (símbolos de salida).
\begin{center}
\begin{tikzpicture}[>={Triangle[width=1.5mm,length=1.5mm]},->,node distance=2cm,auto]
\node[] (q_0) {${ }$};
\node[state,rectangle] (q_1) at (2,0) {$s\in k$};
\node[] (q_2) at (4,0) {${ }$};

\path[->,line width=0.25mm] (q_0) edge node[] {$\sigma\in\Sigma$} (q_1);
\path[->,line width=0.25mm] (q_1) edge node[] {$\delta\in\Delta$} (q_2);
%\path[->] (q_0) edge node[swap] {$\sigma$} (q_1);
\end{tikzpicture}
\end{center}
\subsubsection{Representación}
\begin{itemize}
\item \textbf{Tabla de Transición}
\item \textbf{Grafo} \\

\begin{figure}[h]
\centering
\begin{tikzpicture}[>={Triangle[width=1.5mm,length=1.5mm]},->,node distance=2cm,auto]
\node[state] (q_0) {$s_i$};
\node[state] (q_1) at (3,0) {$s_j$};

\path[->,line width=0.25mm] (q_0) edge node[] {$\sigma/\delta$} (q_1);
%\path[->] (q_0) edge node[swap] {$\sigma$} (q_1);
\end{tikzpicture}
\caption{ssi: $f(s_i,\sigma)=s_j \wedge g(s_i,\sigma)=\delta$ }
\end{figure}
\end{itemize}
\subsubsection{Comportamiento Dinámico}
Sea $M=(k,\Sigma,\Delta,f,g)$ una máquina: \\
\begin{center}
\begin{tikzpicture}[>={Triangle[width=1.5mm,length=1.5mm]},->,node distance=2cm,auto]
\node[] (s_0) {$s_0$};
\node[] (s_1) at (2,0) {$s_1$};
\node[] (s_2) at (4,0) {$s_2$};
\node[] (sus) at (6,0) {$\cdots$};
\node[] (s_k) at (8,0) {$s_k$};

\node[] (t_0) at (0,1) {$t_0$};
\node[] (t_1) at (2,1) {$t_1$};
\node[] (t_2) at (4,1) {$t_2$};
\node[] (tsus) at (6,1) {$\cdots$};
\node[] (t_k) at (8,1) {$t_k$};

\path[->,line width=0.25mm] (s_0) edge node[] {$\sigma_0/\delta_0$} (s_1);
\path[->,line width=0.25mm] (s_1) edge node[] {$\sigma_1/\delta_1$} (s_2);
\path[->,line width=0.25mm] (s_2) edge node[] {$\sigma_2/\delta_2$} (sus);
\path[->,line width=0.25mm] (sus) edge node[] {$\sigma_{k-1}/\delta_{k-1}$} (s_k);
%\path[->] (q_0) edge node[swap] {$\sigma$} (q_1);
\end{tikzpicture}
\end{center}