\section{Media Aritmetica Simple $(\overline{X})$} 
\subsection{Cálculo}
\subsubsection{Para Datos no Agrupados (Originales)}
Sean $x_1,x_2,x_3,\ldots ,x_n$ valores de la variable $X$. La media Arítmetica simple se denota por $M(x),\theta$ o $\bar{X}$ está dada por:
$$M(x)=\bar{X}=\dfrac{\displaystyle\sum_{i=1}^{n}x_i}{n}$$
\subsubsection{Para Agrupados}
$$M(y)=\bar{Y}=\dfrac{\displaystyle\sum_{i=1}^{n}y_i\cdot f_i}{n}$$
\subsection{Propiedades}
\begin{enumerate}
\item En todas las distribuciones de frecuencias, la sumatoria simple de las desviaciones de los valores de la variable respecto a la media aritmética es 0.
\begin{multicols}{2}
\textbf{Datos Originales} 
\begin{center}
$\displaystyle\sum_{i=1}^{n}(x_i-\bar{x})=0$
\end{center}
\columnbreak
\textbf{Datos Agrupados}
$$\displaystyle\sum_{i=1}^{n}(y_i-\bar{y})f_i=0$$
\end{multicols}
\item En toda distribución, la sumatoria de los cuadrados de las desviaciones de los valores de las variables, respecto de la media es mínima
\begin{multicols}{2}
\textbf{Datos Originales}
\begin{center}
$\displaystyle\sum_{i=1}^{n}(x_i-\bar{x})^2 \leq \displaystyle\sum_{i=1}^{n}(x_i-B)^2 $
\end{center}
\columnbreak
\textbf{Datos Agrupados}
$$\displaystyle\sum_{i=1}^{n}(y_i-\bar{y})^2 \leq \displaystyle\sum_{i=1}^{n}(y_i-B)^2\cdot f_i $$ 
\end{multicols}
Para cualquier constante $B$.
\item $M(a x_i)=\bar{ax_i} = a\cdot M(x_i) = a\bar{x}$, $a$ constante
\item $M(a \pm x_i)=\overline{{a \pm x_i}} = a\pm M(x_i) = a\pm\bar{x}$, $a$ constante
\item $M(a)=\bar{a}=a$, $a$ constante
\end{enumerate}
\subsection{Métodos Abreviados para el Cálculo de la Media}
\subsubsection{Primer Método Abreviado}
$$\bar{y}=O_t + \dfrac{\displaystyle\sum_{i=1}^{n}z_i'\cdot f_i}{n}$$
Donde: $z_i'=y_i - O_t$
\subsubsection{Segundo Método Abreviado}
$$\bar{y}=O_t + C\cdot \dfrac{\displaystyle\sum_{i=1}^{n}z_i''\cdot f_i}{n}$$
Donde: $z_i''=\dfrac{y_i - O_t}{C}=\dfrac{z_i'}{C}$\\${ }$\\
Y para ambos métodos: $O_t$ es el origen del trabajo (cualquier valor de la marca de clase).
\subsubsection{Media Aritmética en Distribuciones Simétricas}
\begin{multicols}{2}
Si $k$ es impar
\begin{center}
$\bar{y}=y_{\frac{k+1}{2}}$
\end{center}
\columnbreak
Si $k$ es par
\begin{center}
$\bar{y}=y_{\frac{k}{2}}'$
\end{center}
\end{multicols}
\subsection{Ventajas y Desventajas}
\begin{multicols}{2}
\textbf{Ventajas}
\begin{itemize}
\item Es la medida de tendencia central mas usada.
\item El promedio es estable en el muestreo.
\item Puede ser usado como un detector de variación en los datos.
\item Se emplea en cálculos estadísticos posteriores.
\end{itemize}
\columnbreak
\textbf{Desventajas}
\begin{itemize}
\item Es sensible a los valores extremos.
\item Si el conjunto de datos es muy grande puede ser tedioso su cálculo.
\item No se puede calcular para datos cualitativos.
\item No se puede calcular para datos que tengan clases de extremos abiertos, tanto inferior o superior.
\end{itemize}
\end{multicols}
\subsection{Ejemplos}