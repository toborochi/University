\chapter{Ecuaciones Diferenciales Homogéneas}
\section{Ecuaciones Diferenciales Homogéneas con Coeficientes Constantes}
\begin{equation}
ay'' + by' +cy = 0
\end{equation}
Donde $a,b$ y $c$ son constantes. \\${ }$\\
Ecuaciones de este tipo tienen dos soluciones independientes.
\subsection{Solución General}
En general, si $y=y_1(x)$ y $y=y_2(x)$ son ambas soluciones de $1\texttt{.}8$ entonces, cualquier combinación lineal de $y_1$ y $y_2$ es también solución de $1\texttt{.}8$. Esto quiere decir:
\begin{align*}
 ay_1''(x)+by_1'(x) +cy_1(x)=0 \\
ay_2''(x)+by_2'(x) +cy_2(x)=0
\end{align*}
Sea: $y=Ay_1(x)+By_2(x)$ una combinación lineal de $y_1$ y $y_2$.
\section{Ecuaciones Diferenciales Homogéneas con Coeficientes Constantes de Orden $n$}
$$a_ny^{(n)}+a_{n-1}y^{(n-1)}+\cdots + a_0 y = 0$$
Donde: $a_i$ son constantes y $a_n\neq 0$ \\${ }$\\
El procedimiento es similar al de orden 2. \\${ }$\\
Si $y=e^{mx}$ es una solución, entonces $m$ es una raíz de la Ecuación Característica.
$$\boxed{a_nm^n+a_{n-1}m^{n-1}+\cdots +a_1 m + a_0 = 0}$$
El polinomio de grado $n$ tiene $n$ raíces, pero no necesitan ser distinas, la multiplicidad de una raíz es el número de veces que se repite. La suma de todas las multiplicidades de todas las raíces distintas es igual al grado del polinomio.
\subsection{Procedimiento}