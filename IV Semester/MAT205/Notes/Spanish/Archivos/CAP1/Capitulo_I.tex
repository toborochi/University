\chapter{Aproximaciones y Errores}
\section{Errores}
\begin{itemize}
\item \textbf{Errores de Redondeo:} Se debe a que el computador solo puede representar cantidades con un \textit{número finito }de dígitos.
\item \textbf{Errores de Truncamiento:} Representa la diferencia entre la formulación  matemática exacta de un problema y la aproximación dada por un método numérico.
\item \textbf{Cifras Significativas:} Se refiere a la confiabilidad de un valor numérico. Es el número de dígitos mas un dígito estimado que se puede usar con confianza.\\${ }$\\
$\blacklozenge$ Los ceros no siempre son cifras significativas ya que pueden usarse solo para ubicar el punto decimal. 
\end{itemize}
\section{Exactitud y Precisión}
\begin{itemize}
\item \textbf{Exactitud:} Se refiere a la aproximación de un número o medida al valor verdadero que se supone presenta.
\item \textbf{Precisión:} Se refiere a:
\begin{itemize}
\item Al número de cifras significativas que representa una cantidad.
\item La extensión en las lecturas repetidas, de un instrumento que mide alguna propiedad física.
\end{itemize}
\end{itemize}