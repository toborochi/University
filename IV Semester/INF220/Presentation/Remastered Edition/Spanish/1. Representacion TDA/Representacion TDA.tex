\documentclass[11pt]{beamer}
\usetheme{Warsaw}
\usepackage[utf8]{inputenc}
\usepackage[spanish]{babel}
\usepackage{amsmath}
\usepackage{amsfonts}
\usepackage{amssymb}
\usepackage{graphicx}
\author{M. L. Winnipeg\inst{1}}
\title{Representación TDA}
\graphicspath{ {./images/} }
%\setbeamercovered{transparent} 
%\setbeamertemplate{navigation symbols}{} 
\logo{\includegraphics[height=1.5cm]{ficctlogo.png}} 
\institute[UAGRM]{

  \inst{1}%
  Facultad de Ingeniería en Ciencias de la Computación y Telecomunicaciones\\
  Universidad Autónoma Gabriél René Moreno
} 
%\date{} 
\subject{Estructura de Datos I} 
\begin{document}

\begin{frame}
\titlepage
\end{frame}

\begin{frame}
\tableofcontents
\end{frame}

\begin{frame}{Introducción}
\begin{block}{Objetivo Central}
Aplicar los conceptos de estructuras de datos y sus algoritmos de manipulación, para la implementación de estructuras de datos clásicas y creación de nuevas estructuras en la solución de problemas.
\end{block}
\end{frame}


\begin{frame}{Problemas, programas, algoritmos y Estructuras de Datos}
\begin{alertblock}{Afrontando un Problema usando TDAs}
Problema $\rightarrow$ Algoritmos + Estructuras de Datos $\rightarrow$ Programa
\end{alertblock}
\pause
\begin{itemize}
\item<1-> \textbf{Problema:} Conjuntos de hechos o circunstancias que dificultan la realización de un fin.
\item<2-> Algoritmos:
\item<3-> \alert{Estructura de Dato:}
\end{itemize}

\end{frame}

\end{document}