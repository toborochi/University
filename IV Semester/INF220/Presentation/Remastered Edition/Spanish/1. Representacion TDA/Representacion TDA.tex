\documentclass[11pt]{beamer}
\usetheme{Warsaw}
\usepackage[utf8]{inputenc}
\usepackage[spanish]{babel}
\usepackage{amsmath}
\usepackage{amsfonts}
\usepackage{amssymb}
\usepackage{graphicx}
\author{M. L. Winnipeg\inst{1}}
\title{Representación TDA}
\graphicspath{ {./images/} }
%\setbeamercovered{transparent} 
%\setbeamertemplate{navigation symbols}{} 
\logo{\includegraphics[height=1.5cm]{ficctlogo.png}} 
\institute[UAGRM]{
  \inst{1}%
  Facultad de Ingeniería en Ciencias de la Computación y Telecomunicaciones\\
  Universidad Autónoma Gabriél René Moreno
} 
%\date{} 
\subject{Estructura de Datos I} 
\begin{document}

\begin{frame}
\titlepage
\end{frame}

\begin{frame}
\tableofcontents
\end{frame}

\section{Introducción}
\begin{frame}{Introducción}
\begin{block}{Objetivo Central}
Aplicar los conceptos de estructuras de datos y sus algoritmos de manipulación, para la implementación de estructuras de datos clásicas y creación de nuevas estructuras en la solución de problemas.
\end{block}
\end{frame}

\subsection{Problemas, Programas, Algoritmos y Estructuras de Datos}
\begin{frame}{Problemas, Programas, Algoritmos y Estructuras de Datos}
\begin{alertblock}{Afrontando un Problema usando TDAs}
Problema $\rightarrow$ Algoritmos + Estructuras de Datos $\rightarrow$ Programa
\end{alertblock}
\pause
\begin{itemize}[<+->]
\item
 Problema: Conjuntos de hechos o circunstancias que dificultan la realización de un fin. \\
\item
Algoritmos: Conjunto de reglas finitas y sin ambiguedad. \\
\item
\alert{Estructura de Dato: Disposición en memoria de los datos.
}
\item
Programa: Algoritmos + Estructuras de Datos.
\end{itemize}

\end{frame}
\section{Abstracción}
\subsection{Abstracción de Datos}
\begin{frame}{Abstracción}
¿Qué es la Abstracción de Datos?
\begin{itemize}[<+->]
\item \textbf{La abstracción de datos} es una técnica o
metodología que permite diseñar estructuras
de datos.
\item Consiste básicamente en representar bajo
ciertos lineamientos de formato las
características esenciales de una estructura
de datos.
\item Este proceso de diseño se olvida de los
detalles específicos de \textbf{implementación} de
los datos.
\end{itemize}
\end{frame}
\subsection{Usos de la Abstracción}
\begin{frame}{Usos de la Abstracción}
¿En que se usa la Abstracción?
\begin{itemize}[<+->]
\item Procedimientos y funciones son \textbf{abstracciones de control}.
\item Los tipos definidos por el usuario son \textbf{abstracciones de datos}.
\item Las unidades, módulos o paquetes son abstracciones de nivel superior: \textbf{abstracciones de funcionalidades}.
\end{itemize}
\end{frame}
\subsection{Conclusiones}
\begin{frame}
Conclusiones
\begin{itemize}[<+->]
\item Es una técnica poderosa de programación que permite inventar o definir nuevos tipos de datos observando e identificando entidades del mundo real (objetos) ocultando datos irrelevantes para la solución del problema.
\item Gracias a esta técnica se pueden diseñar programas mas cortos, legibles y flexibles.
\item Estos nuevos tipos de datos se conocen como \alert{Tipo de Datos Abstractos}
\end{itemize}
\end{frame}

\section{Estructuras de Datos}

\begin{frame}{Estructura de Datos}
¿Que es una Estructura de Datos? \pause
\begin{block}{Estructura de Datos}
Una estructura de datos es básicamente un
grupo de elementos de datos que se agrupan
bajo un nombre, y que define una forma
particular de almacenar y organizar datos en
una computadora para que pueda ser utilizada
eficientemente. \\
\begin{flushright}
{\scriptsize \textit{Data Estructures using C,} Reema Thareja (Pág. 43 último párrafo).}
\end{flushright}
\end{block}
\end{frame}
\subsection{Clasificación}
\begin{frame}{Clasificación de Estructuras de Datos}
\begin{itemize}
\item Estructuras de Datos \alert{Primitivas}: son los tipos de datos fundamentales que son compatibles con una programación. Algunos tipos de datos básicos son enteros, reales, caracteres y booleanos.
\pause
\item Estructuras de Datos \alert{no Primitivas}: son aquellas
estructuras de datos que se crean utilizando datos primitivos.
\pause
\begin{itemize}
\item \alert{Lineales:} Si los elementos de una estructura de datos se almacenan en
un orden lineal o secuencial, entonces pertenece a la categoría lineal.
\pause
\item \alert{No Lineales:} Si los elementos de una estructura de datos no se
almacenan en un orden secuencial, entonces es una estructura de
datos no lineal.
\end{itemize}
\end{itemize}
\end{frame}

\section{Tipos de Datos Abstractos}
\subsection{Tipos}
\begin{frame}{Tipos de Datos Abstractos}
Es la representación de una entidad u objeto para facilitar su programación. Se compone de:
\begin{itemize}
\item \alert{Estructura de Datos:} Es la estructura de programación que se selecciona para representar las características de la entidad modelada.
\pause
\item \alert{Funciones de Abstracción:} Son funciones que permiten hacer uso de la estructura de datos, y que esconden los detalles de dicha estructura, permitiendo un mayor nivel de abstracción.
\end{itemize}
\end{frame}
\section{Formas de Implementación}
\begin{frame}

\end{frame}

\subsection{Modelo Estático}
\begin{frame}

\end{frame}
\subsection{Modelo Dinámico}
\subsection{Modelo Persistente}
\subsection{Modelo Simulado}
\end{document}