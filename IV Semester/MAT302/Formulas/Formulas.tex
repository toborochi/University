\documentclass[10pt,letterpaper]{article}
\usepackage[utf8]{inputenc}
\usepackage[spanish]{babel}
\usepackage{amsmath}
\usepackage{amsfonts}
\usepackage{amssymb}
\usepackage{graphicx}
\usepackage{subfiles}
\usepackage{enumitem}
\usepackage[affil-it]{authblk}
\usepackage{array}
\usepackage{tabularx}
\usepackage{multicol}
\usepackage{slashbox}
\usepackage{diagbox}
\usepackage{slashbox,multirow}
\usepackage{enumitem}
\usepackage{mathtools}
\usepackage{pstricks}
\usepackage{pgfplots}
\usepackage{tikz}
\usepackage{calc}
\usepackage{ifthen}
\usepackage{mathrsfs} %Contiene el Signo de Transformada de Laplace
\usepackage{empheq}
\usepackage{svg}
\usepackage{hyperref}
\usepackage[left=2cm,right=2cm,top=2cm,bottom=2cm]{geometry}
\author{Leonardo H. Añez Vladimirovna}
\title{Formulas de Estadística Inferencial}
\begin{document}
\maketitle

\section{Variables Aleatorias}
\begin{multicols}{2}
\begin{flushleft}
\subsection{Variables Aleatorias Discretas}
Notación:  $P(A),P(X=x),f(x)$.
\begin{enumerate}
\item $p(x)\geq 0 ; \forall x \in \mathbb{R}$
\item $\displaystyle\sum_{x_i\in Rec_x} p(x_i)=1$
\item $\displaystyle\sum_{i=1}^{n} p(x_i)=1$
\item $\displaystyle\sum_{i=1}^{\infty} p(x_i)=1$
\end{enumerate}
\end{flushleft}
\columnbreak
\subsection{Variables Aleatorias Continuas}
Notación:  $F(x),P(X\leq x)$.
\begin{enumerate}
\item $f(x)\geq 0;\forall x \in \mathbb{R}$
\item $\displaystyle\int_{-\infty}^{+\infty} f(x)dx = 1$
\item $P(a\leq x \leq b) = \displaystyle\int_{a}^{b} f(x)dx=1$
\end{enumerate}
\end{multicols}

\end{document}