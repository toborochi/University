\documentclass[10pt,letterpaper]{article}
\usepackage[utf8]{inputenc}
\usepackage[spanish]{babel}
\usepackage{amsmath}
\usepackage{amsfonts}
\usepackage{amssymb}
\usepackage{graphicx}
\usepackage{subfiles}
\usepackage{enumitem}
\usepackage[affil-it]{authblk}
\usepackage{array}
\usepackage{tabularx}
\usepackage{multicol}
\usepackage{slashbox}
\usepackage{diagbox}
\usepackage{slashbox,multirow}
\usepackage{enumitem}
\usepackage{mathtools}
\usepackage{pstricks}
\usepackage{pgfplots}
\usepackage{tikz}
\usepackage{calc}
\usepackage{ifthen}
\usepackage{mathrsfs} %Contiene el Signo de Transformada de Laplace
\usepackage{empheq}
\usepackage{svg}
\usepackage{hyperref}
\usepackage[left=2cm,right=2cm,top=2cm,bottom=2cm]{geometry}
\author{Leonardo H. Añez Vladimirovna}
\title{Formulas de Estadística Inferencial \texttt{(MAT302)}}
 \pgfplotsset{compat=1.16}
\begin{document}
\maketitle

\section{Variables Aleatorias}
\subsection{Definiciones}
\begin{multicols}{2}
\subsubsection{Discretas}
\begin{flushleft}
Notación:  $P(A),P(X=x),f(x)$.
\begin{enumerate}
\item $p(x)\geq 0 ; \forall x \in \mathbb{R}$
\item $\displaystyle\sum_{x_i\in Rec_x} p(x_i)=1$
\item $\displaystyle\sum_{i=1}^{n} p(x_i)=1$
\item $\displaystyle\sum_{i=1}^{\infty} p(x_i)=1$
\end{enumerate}
\end{flushleft}
\columnbreak
\subsubsection{Continuas}
\begin{flushleft}
Notación:  $F(x),P(X\leq x)$.
\begin{enumerate}
\item $f(x)\geq 0;\forall x \in \mathbb{R}$
\item $\displaystyle\int_{-\infty}^{+\infty} f(x)dx = 1$
\item $P(a\leq x \leq b) = \displaystyle\int_{a}^{b} f(x)dx=1$
\end{enumerate}
\end{flushleft}
\end{multicols}
\subsection{Propiedades}
\begin{multicols}{2}
\subsubsection{Discreta}
\begin{enumerate}
\item $0\leq F(x) \leq 1, \forall x\in \mathbb{R}$
\item $F(-\infty)=0$
\item $F(+\infty)=1$
\item $P(X\leq a)=F(a)$
\item $P(X>a)=1-P(X\leq a)=1-F(a)$
\item $P(X<a)=
\begin{cases}
F(a-1); a\in\mathbb{Z} \\
F( [\![ x ]\!] ) ; a \notin \mathbb{Z}
\end{cases}$
\item $P(X\leq -a)=1-P(x\leq a)=1-F(a)$
\item $P(a<x\leq b)=F(b)-F(a)$
\item $P(X\leq x\leq)=F(b)-F(a)+P(X=a)$
\item $P(a<x<b)=F(b)-F(a)-P(X=b)$
\item $P(X=x_i)=F(x_i)-F(x_{i-1})$
\end{enumerate}
\columnbreak
\subsubsection{Continua}
\begin{enumerate}
\item $0\leq F(x) \leq 1, \forall x\in \mathbb{R}$
\item $F(-\infty)=0$
\item $F(+\infty)=1$
\item $P(X\leq a)=P(X<a)=F(a)$
\item $P(X>a)=1-P(X\leq a)=1-F(a)$
\item $P(X\geq a)=1-P(X<a)=1-F(a)$
\item $P(X\leq -a)=1-P(X\leq a)=1-F(a)$
\item $P(a<X\leq b)=P(a\leq X\leq b)=P(a<X<b)=F(b)-F(a)$
\item $f(x)=\dfrac{dF(x)}{dx}$
\end{enumerate}
\end{multicols}
\subsection{Esperanza}
\begin{multicols}{3}
\subsubsection{V.A.s Discretas}
$E(x)=\mu=\mu_x=\displaystyle\sum_x x\cdot p(x)$ \\ \vspace{0.05cm} \\
$X$ v.a. con función $f$: \\ \vspace{0.05cm} \\
$E(g(x))=\mu_{g(x)}=\displaystyle\sum_x g(x)\cdot f(x)$
\columnbreak
\subsubsection{V.A.s Continua}
$E(x)=\displaystyle\int_{-\infty}^{+\infty} xf(x) dx$\\ \vspace{0.05cm} \\
$X$ v.a. con función $f$: \\ \vspace{0.05cm} \\
$E(x)=\displaystyle\int_{-\infty}^{+\infty} g(x)\cdot f(x) dx$
\columnbreak
\subsubsection{Propiedades}
$\bigstar$ {\scriptsize $a$ y $b$ constantes.}
\begin{enumerate}
\item $E(a)=a$
\item $E(x\pm a)=E(x)\pm a$
\item $E(ax)=aE(x)$
\item $E(ax\pm b)=aE(x)\pm b$
\end{enumerate}
\end{multicols}



\subsection{Varianza}
\begin{multicols}{3}
\subsubsection{V.A.s Discretas}
\begin{align*}
V(x)=\sigma^2 &=  E(x-\mu)^2 \\
& = \displaystyle\sum_x (x-\mu)^2 f(x) 
\end{align*}
\columnbreak
\subsubsection{V.A.s Continua}
\begin{align*}
V(x)=\sigma^2 &=  E(x-\mu)^2 \\
& = \displaystyle\int_{-\infty}^{+\infty} (x-\mu)^2 f(x) dx
\end{align*}
\columnbreak
\subsubsection{Propiedades}
\begin{enumerate}
\item $V(x)\geq 0$
\item $V(a)=0$
\item $V(ax)=aV(x)$
\item $V(ax\pm b)=a^2 V(x)\pm b$
\item $V(x)=E(x^2)-[E(x)]^2$
\end{enumerate}
\end{multicols}
\subsection{Función de Probabilidad Conjunta}
\begin{multicols}{2}
\subsubsection{Función de Cuantía Conjunta}
\begin{enumerate}
\item $P(x,y)=P(X=x,Y=y)\geq 0$
\item $\displaystyle\sum_x \displaystyle\sum_y P(x,y)=1$
\item $P((x,y)\in A) = \sum_A \sum P(x,y)$
\end{enumerate}
\columnbreak
\subsubsection{Función de Densidad Conjunta}
\end{multicols}

\subsection{Distribuciones Marginales}
\subsection{Covarianza}
\subsection{Resultados Importantes}

\section{Distribuciones}
\end{document}