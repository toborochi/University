\chapter{Modelos de Distribución de Probabilidad}
\section{Distribución de Bernoulli}
Si la probabilidad de que ocurra un evento $p$ y la probabilidad de que no ocurra es $q(q=1-p)$, entonces se dice que la \textbf{v.a.} discreta $X$ se distribuye según Bernoulli, cuya función de cuantía está dada por:
$$p(x)=f(x)=P(X=x)=
\begin{cases}
p^x\cdot (1-p)^{1-x} &; x=0;1 \\
0 &; \text{otro caso}
\end{cases}
$$
Y cuya función de distribución acumulada es:
$$
F(X)=P(X=x)=
\begin{cases}
0 &; x<0\\
q=1-p &; 0\leq x < 1\\
1 &; x\geq 1
\end{cases}
$$

$$
X \sim Ber(x;p)\Rightarrow 
\begin{cases}
E(x)=\mu = p \\
V(x)=\sigma^2 = p\cdot q \\
D(x) = \sigma = \sqrt{p\cdot q}
\end{cases}
$$
Conocida como prueba o ensayo de Bernoulli, es un experimento que solo tiene 2 resultados posibles, a los cuales se los llama:
\begin{itemize}
\item Éxito $(p)$
\item Fracaso $(q)$
\end{itemize}
\section{Distribución Binomial}
Una \textbf{v.a.} discreta $X$ tiene distribución lineal si su función de cuantía está dada por:
$$
f(x)=
\begin{cases}
\displaystyle\binom{n}{x} p^x \cdot q^{n-x} &; x=0,1,2,\ldots,n \\
0 &; \text{otro caso}
\end{cases}
$$