\chapter{Variables Aleatorias}
Una variable aleatoria $x$ (desde ahora denotada por \textbf{v.a.}) es una función definida sobre el espacio muestral $S$ con valores en $\mathbb{R}$ que a cada elemento de $S$ (Punto muestral) hace corresponder un número real $x=X$.

$$x=X(w) \in Rec_X \subseteq \mathbb{R}$$ 
\subsubsection{Gráficamente}
%Pendiente
\subsubsection{Notación Conjuntista}
$$X = \lbrace (w,x)\ w\in S, x=X(w)\in \mathbb{R} \rbrace \subseteq S\times  \mathbb{R}$$
Donde:
\begin{itemize}
\item $S$: Conjunto Partida (Espacio Muestral).
\item $\mathbb{R}$: Conjunto de llegada.
\item $w$: Elemento de $S$ (Punto Muestral).
\item $x$: Valor de la \textbf{v.a.} $X$.
\item $Rec_X$: Recorrido de $X$.
\item $X$: Función \textbf{v.a.} (Conjunto de Pares Ordenados).
\end{itemize}
\subsubsection{Notaciones}
Las \textbf{v.a.} se denotan con letras mayúsculas tales como $X,Y$ o $Z$, y los valores correspondientes con letras minúsculas.
\section{Clasificación de Variables Aleatorias}